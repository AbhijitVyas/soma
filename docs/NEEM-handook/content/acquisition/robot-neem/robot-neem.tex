\section{Robot NEEMs}
\label{sec:robot-neem}
\lstset{style=lispcode}
This section describes how to generate \neems from experiments performed by the robot either in simulation or in the real world. 

\subsection{Prerequisite}

Before you are intending to generate your episodic memories, make sure you are familiar with the Cognitive Robot Abstract Machine (\cram)\footnote{\url{http://cram-system.org/cram}} system and it is installed on your machine.
\cram is a planning framework which allows to design high-level plan for robots.
This section requires that your robot plan is written in \cram to be able to generate \neems.
However, once you are familiar with our planning and logging components, you will be able to port those components to your preferred planning tool.

To generate \neems  you will need also the following software components to be installed:

\begin{itemize}
	\item A MongoDB server with at least version 3.4.10\footnote{\url{https://www.mongodb.com/}}
	\item \knowrob\footnote{\url{https://github.com/knowrob/knowrob}}
	\item \cram ontology\footnote{\url{https://github.com/ease-crc/cram\_knowledge}}
\end{itemize}


\subsection{Recording Narrative Enabled Episodic Memories}
Our recording mechanism  captures every executed \cram action and its parameter.
In addition, the logger puts the actions in relation to each other by creating a hierarchy which is described in Section \ref{sec:composition}.

Before you can begin to record your own \neems, you need to include the "cram-cloud-logger" package into your \cram plan.
This package is by default included with your \cram installation.
To start and finish the logging process, run \textit{ccl::start-episode} before your plan execution and when you are finished with your experiment run \textit{ccl::stop-episode}.
It can look like the following:
\begin{lstlisting}[language=lisp, caption=Steps to Record an Episode for a \cram Plan]
	(ccl::start-episode)
	(urdf-proj:with-simulated-robot (demo::demo-random nil ))
	(ccl::stop-episode)
\end{lstlisting}
	
The generated \neem will be stored per default in "\raisebox{-0.9ex}{\~{}}/knowrob-memory".
Keep in mind to  have \knowrob launched before starting the \neem recording. 
You can start \knowrob via:

\begin{lstlisting}[language=bash, caption=How to Start \knowrob]
	roslaunch knowrob_memory knowrob.launch
\end{lstlisting}

By default your \neem will be stored under "\raisebox{-0.9ex}{\~{}}/knowrob-memory/\textless timestamp \textgreater".


\subsection{Add Semantic Support to your Designed Plans}
The disadvantage of having a strong semantic knowledge representation is that the used ontology requires updates when semantic knowledge has to be extended. 
Currently, the \ease project focuses on the support to record table setting/cleaning-up experiments.
If you want to create \neems for e.g.\ autonomous cars, you will need to extend the \soma ontology and the recording mechanism with your required actions, parameters and objects.
How to extend the \soma ontology is described in Section \ref{sec:extension}\todo{Seba: Fix reference to ontology section}.
In the following subsection, we will describe how you can extend the \cram recording mechanism to support your required semantic knowledge.

\paragraph{New Tasks Definition}
If you want to semantically record new tasks such as e.g. accelerating or breaking, you need to define them in the ontology as described in Section \ref{sec:extension,subsec:task}\todo{Seba: Set extension reference}.
Tasks which are not defined in the ontology, will be logged as \textit{PhysicalTask}.
The \textit{PlanExecution} instance pointing to the \textit{PhysicalTask} will have a comment attached with the statement "Unknown Action: \textless CRAM-ACTION-NAME\textgreater", where \textless CRAM-ACTION-NAME\textgreater is the action name you defined in your plan.
After you defined your new actions in \soma, please open the "knowrob-action-name-handler.lisp" in the "cram-cloud-logger" package and add your new actions in the format:

\begin{lstlisting}[language=lisp, caption=Linking the CRAM Action to the Ontology Concept]
(defun init-action-name-mapper ()
   (let ((action-name-mapper (make-instance 'ccl::cram-2-knowrob-mapper))
     (definition
         '(("reaching" "Reaching")
           ("retracting" "Retracting")
           ("lifting" "Lifting")
           ("opening" "Opening")
           ("<CRAM-ACTION-NAME>" "<SOMA-ONTOLOGY-NAME>")
\end{lstlisting}

where \textless CRAM-ACTION-NAME\textgreater~is the name of your action used in the cram plan and \textless SOMA-ONTOLOGY-NAME\textgreater~is the concept name of the task defined in \soma. 
With this step you added successfully the new action to the \cram \neem episodic memory logger.

\paragraph{Adding New Objects}
Unknown objects will be logged as instance of \textit{DesignedArtifact}.
In addition, a comment like "Unknown Object: \cramentitytoreplace{CRAM-OBJECT-TYPE}" will be attached to this instance.
\cramentitytoreplace{CRAM-OBJECT-TYPE} indicates which object you need to define in the \soma ontology.
After you added your object to the ontology, open the "utils-for-perform.lisp" in the \cramloggerpackage package and include the new object in the hash table generate in "get-ease-object-lookup-table" like:

\begin{lstlisting}[language=lisp, caption=Linking the CRAM Object to the Ontology Concept]
(defun get-ease-object-lookup-table()
  (let ((lookup-table (make-hash-table :test 'equal)))
    (setf (gethash "BOWL" lookup-table) "'http://www.ease-crc.org/ont/SOMA.owl#Bowl'")
    (setf (gethash "CUP" lookup-table) "'http://www.ease-crc.org/ont/SOMA.owl#Cup'")
    (setf (gethash "<CRAM-OBJECT-TYPE>" lookup-table) "'<SOMA-ONTOLOGY-ENTITY-IRI>'")
 lookup-table))
\end{lstlisting}

where the key is \cramentitytoreplace{CRAM-OBJECT-TYPE}, the object type used in the \cram plan, and \cramentitytoreplace{SOMA-ONTOLOGY-ENTITY-URI}the value which is the uri pointing to the object concept created in \soma.

\paragraph{Adding New Failure}
Unknown \cram failures will be logged as instance of \textit{Failure}.
In addition, a comment like "Unknown failure: \cramentitytoreplace{CRAM-FAILURE-NAME}" will be attached to the instance.
\cramentitytoreplace{CRAM-FAILURE-NAME} indicates which failure you need to define in the ontology.
After you added your failure to the ontology, open the "failure-handler.lisp" in the \cramloggerpackage package and add your new failure in the format:

\begin{lstlisting}[language=lisp, caption=Linking the CRAM Action to the Ontology Concept]
(defun init-failure-mapper ()
  (let ((failure-mapper (make-instance 'ccl::cram-2-knowrob-mapper))
    (definition
      '(("cram-common-failures:low-level-failure" "LowLevelFailure")
        ("cram-common-failures:actionlib-action-timed-out" "ActionlibTimeout")
        ("<CRAM-FAILURE-NAME>" "<SOMA-ONTOLOGY-NAME>")
\end{lstlisting}

where \cramentitytoreplace{CRAM-FAILURE-NAME} is the name of your failure defined in the cram plan and \cramentitytoreplace{SOMA-ONTOLOGY-NAME}.
With this step you added successfully the support of the new failure to \cram \neem episodic memory logger.".	

%\paragraph{Adding New Rostopic}
%Per default, we log the rostopics \tf and tf\_static.
%\todo{Seba: This part is outdated. Check how the new topics can be added now.}
%If you need to log additional topics, open "memory.pl" in the "knowrob\_memory" package from \knowrob and include your topic in the "mem\_episode\_start(Episode)" function.
%After the \neem generation, the data will be stored in the created \neem folder under the file "roslog/\textless rostopic\textgreater.bson" \todo{Abhijit: text is not formatted correctly}.

%\paragraph{Adding New Parameters}
%Unknown parameters will be logged as comment attached to the corresponding \textit{PlanExecution} instance.
%The comment statement "Unknown Parameter: PARAMETER-NAME  -\#\#\#\#- PARAMETER-VALUE/>"
%The current parameter types are represented 
%
%\todo{@Ontology group: Please make sure that is available in the ontology }
%
%\begin{enumerate} 
%	\item Integer/Floats
%	\item Posen
%	\item Spatial Relations
%	\item Link to entities of other ontologies such as http://knowrob.org/kb/PR2.owl\#pr2\_right\_arm
%\end{enumerate}
%
%Before you want to model your parameter what data type your parameter is.
%If it is a complex object, you need to consider how you want to to represent it an the ontology.
%For simpler representation such has a discrete domain representation, you might represented as the domain values as \textit{Region} and add the model the parameter as a subconcept of \textit{Parameter}.
%More information about the concepts \textit{Region} and \textit{Parameter} can be found in \todo{ reference to Region and parameter}.
%
%
%\subsection{Adding New Reasoning Tasks}
%\todo{@Ontology group: How to log the result of the reasoning query ?}



%\subsection{Next steps}
%\todo{Seba: Is outdated.}
%After you have generated your \neem, you can use the tool \todo{Add neem2narrative} to generate a csv file for your \neem.
%Keep in mind that the csv is a abstraction of \neemnar and can be used to make data-mining on explicit knowledge.
%For more sophisticated analysis, you will need to use \knowrob \todo{Abhijit: an example would be great.}. 
%We use this general analysis to identify bottlenecks in our plan execution \todo{Abhijit: an example would be great.}.
%We also showed that with a collection of \neems we are able to improve the robot's performance.
%\todo{Af}
%The tools for the feature extraction can be found here \todo{Add link}
%Now, we hope that this encourage you to generate your \neems and share them via our \neemhub.

\chapter{Example}
	\label{ch:example}
	\todo{Add belief state example}
	\todo{What about reference to semantic map ?}
	In this chapter we will show a generated NEEM based on the version \neemversion.
	You can download the log used in this example here.	\todo{Provide link for downloading real log file}
	We will not go into the details about the \neemexp meaning the \tf since we just stored the exact \tf message in the database.
	Therefore it is not required to have detail explanation since the idea of \tf is already understood.
	
	This log file represents a experiment where the \pr tired to grasp five objects which were located on a kitchen counter and bring each object to our table and place them there.
	We defined grapsing + placing as a transporting.
	This experiment was performed in projection meaning in simulation.
	\cram is not used using a belief state during execution in projection, therefore we will not show how a belief state is represented.
	The object which were precived or grapsed are represent as strings.
	
	Figure \ref{fig:actionExample} show how a transporting task for a cup is represented.
	To summarize what happend during the transporitng task in the experiment,
	The task was performed not successfully in projection. 
	The failure was that the cup was unfetchable for the \pr.
	The \pr used the left arm to grasp the cup from a connected space region and transported to object to a specific pose in the map.
	\todo{How we can differ between those two goal Locations ?}
	This task has also one sub action. 
	Given the task the agent inferred that to achieve this task, it has to perform a PickingUpAnObject task.
	You can also read from the log that the \pr2 tried already an transporting task on another object and that this task is also not the last transporitng task.
	
	\begin{minipage}{\textwidth}
		\scriptsize
		\begin{lstlisting}[frame=single]
		Individual: TRANSPORTING_KVRJlsOG
		Facts: 
		goalLocation knowrob;ConnectedSpaceRegion_SgfniCWn
		goalLocation knowrob;Pose_eLrKlpcE
		arm pr2;pr2_left_arm
		endTime knowrob;timepoint_1526640320.499945
		failure CRAM-COMMON-FAILURES:OBJECT-UNFETCHABLE
		nextAction knowrob;TRANSPORTING_ysHEUIwW
		objectType "CUP"\todo{add figure to connected space region}
		performedInProjection true
		previousAction knowrob;TRANSPORTING_FkRlsFGa
		startTime knowrob;timepoint_1526640316.535484
		subAction knowrob;PickingUpAnObject_iLTYaxBE
		taskContext TableSetting
		taskSuccess false
		\end{lstlisting}

		\captionof{figure}{Example for a Transporting Task}
		\label{fig:actionExample}
	\end{minipage}
			\vspace{0.5mm}	
	
	The next transporting represented in \dots \ref{fig:actionExample2} represents a transporting task of a bowl object.
	In contrast to the previous task, this task was succuessfully performed.
	Again the \pr used only the left arm.
	The \pr grasped the object from a connected space region which is an kitchen counter \ref{fig:connectedSpaceRegion}.
	It placed the object in the specific pose in the map which is displayed in \ref{fig:pose}.
	

\begin{minipage}{\textwidth}
\scriptsize
\begin{lstlisting}[frame=single]
  Individual: TRANSPORTING_ysHEUIwW
	Facts: 
	  goalLocation knowrob;ConnectedSpaceRegion_YOXaigqs
	  goalLocation knowrob;Pose_GIcRIGdP
	  subAction knowrob;PickingUpAnObject_CTGdupxa
      subAction knowrob;PuttingDownAnObject_uBRtvLcg
      arm pr2;pr2_right_arm
      endTime knowrob;timepoint_1526640332.733415
      nextAction knowrob;TRANSPORTING_HRgQoeEw
      objectType rdf:resource="BOWL"
      performedInProjection true
      previousAction knowrob;TRANSPORTING_KVRJlsOG
      startTime rdf:resource="&knowrob;timepoint_1526640320.561752
	  taskContext "TableSetting"
	  taskSuccess true
\end{lstlisting}
\vspace{2mm}
\captionof{figure}{Example for a second Transporting Task}
\label{fig:actionExample2}
\end{minipage}


\begin{minipage}{\textwidth}
	\scriptsize
\begin{lstlisting}[frame=single]
  Individual: ConnectedSpaceRegion_YOXaigqs
	Facts: 
	  onPhysical knowrob:iai_kitchen_sink_area_counter_top
\end{lstlisting}
	\vspace{2mm}
	\captionof{figure}{Example for Connected Space Region}
	\label{fig:connectedSpaceRegion}
\end{minipage}


\begin{minipage}{\textwidth}
	\scriptsize
\begin{lstlisting}[frame=single]
  Individual: Pose_GIcRIGdP
    Facts: 
	  quaternion 0.0 0.0 1.0 0.0
	  translation -0.7599999904632568 1.190000057220459 0.9300000071525574
\end{lstlisting}
	\vspace{2mm}
	\captionof{figure}{Example for a Pose}
	\label{fig:pose}
\end{minipage}
