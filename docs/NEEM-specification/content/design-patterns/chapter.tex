
% % % % % % % % % % % % % % % % % % % % % % % %
% % % Prelude
% % % % % % % % % % % % % % % % % % % % % % % %
\newcommand{\givenODPNAME}{}
\newcommand{\givenODPINTENT}{}
\newcommand{\givenODPDEFINEDIN}{}
\newcommand{\givenODPDESCRIPTION}{}
\newcommand{\givenODPGRAPHIC}{}
\newcommand{\givenODPDOMAIN}{}
\newcommand{\givenODPQUESTION}{}
\newcommand{\ODPINTENT}[1]     {\renewcommand{\givenODPINTENT}{#1}}
\newcommand{\ODPDEFINEDIN}[1]  {\renewcommand{\givenODPDEFINEDIN}{#1}}
\newcommand{\ODPDESCRIPTION}[1]{\renewcommand{\givenODPDESCRIPTION}{#1}}
\newcommand{\ODPGRAPHIC}[1]    {\renewcommand{\givenODPGRAPHIC}{#1}}
\newcommand{\ODPDOMAIN}[1]     {\renewcommand{\givenODPDOMAIN}{#1}}
\newcommand{\ODPQUESTION}[1]   {\renewcommand{\givenODPQUESTION}{#1}}
\newcommand{\OPDinit}{
  \renewcommand{\givenODPINTENT}{REQUIRED!}
  \renewcommand{\givenODPDEFINEDIN}{REQUIRED!}
  \renewcommand{\givenODPDESCRIPTION}{REQUIRED!}
  \renewcommand{\givenODPGRAPHIC}{REQUIRED!}
  \renewcommand{\givenODPQUESTION}{}
  \renewcommand{\givenODPDOMAIN}{}
  \renewcommand{\labelitemi}{$\mathbf{\sqsubseteq}$}
}

\newenvironment{owlclass}[2][,] {
  \begin{minipage}{5.0cm}
  \begin{center}
  \texttt{\bf#2} \\[-0.2cm]
  \par\noindent\rule{\textwidth}{0.4pt}
  \vspace{-0.6cm}
  \begin{itemize}[#1]
  \raggedright} {
  % % % % % %
  \end{itemize}
  \end{center}
  \end{minipage}
}

\newenvironment{ODP}[1]{
\OPDinit
\renewcommand{\givenODPNAME}{#1}
}{
\givenODPDESCRIPTION
\begin{figure}[htb]
\begin{minipage}{0.45\textwidth}
\begin{tabular}{ p{1.8cm} p{3.2cm} }
\toprule
% {\it\bf Name}                 & \emph{\givenODPNAME} \\
{\it\bf Intent}               & \givenODPINTENT \\
{\it\bf Domains}              & \givenODPDOMAIN \\
{\it\bf Competency Questions} & \givenODPQUESTION \\
{\it\bf Defined in}           & \givenODPDEFINEDIN \\
\bottomrule
\end{tabular}
\end{minipage}
\begin{minipage}{0.55\textwidth}
\begin{center}
\givenODPGRAPHIC
\end{center}
\end{minipage}
\caption{The \emph{\givenODPNAME} ODP.}
\end{figure}
}

\tikzset{owlclass/.style={draw=blue!40,fill=blue!20,rounded corners}}

% % % % % % % % % % % % % % % % % % % % % % % %
% % % % % % % % % % % % % % % % % % % % % % % %
% % % % % % % % % % % % % % % % % % % % % % % %
\chapter{Ontology Design Patterns (ODPs)}

% % % % % % % % % % % % % % % % % % % % % % % %
% \section{Object Roles}

% % % % % % % % % % % % % % % % % % % % % % % %
% % % % % % % % % % % % % % % % % % % % % % % %
% \section{Quality and Quality Regions}

% % % % % % % % % % % % % % % % % % % % % % % %
% % % % % % % % % % % % % % % % % % % % % % % %
% \section{Designed Artifacts}

% % % % % % % % % % % % % % % % % % % % % % % %
% % % % % % % % % % % % % % % % % % % % % % % %
\section{Task Execution ODP}
\begin{ODP}{Task Execution}
\ODPDESCRIPTION{
This ODP allows to make assertions on roles played by agents
without involving the agents that play that roles, and vice versa.
It allows to express neither the context type in which tasks are defined,
not the particular context in which the action is carried out.
Moreover, it does not allow to express the time at which
the task is executed through the action
(for actions that do not solely execute that certain task).}
\ODPINTENT{To represent actions through which tasks are executed. }
\ODPDOMAIN{
  \texttt{Organization},
  \texttt{Management},
  \texttt{Scheduling},
  \texttt{Workflow}}
\ODPDEFINEDIN{DUL.owl}
\ODPQUESTION{
  \emph{Which task is executed through this action?}
  \emph{What actions can execute that task?}}
\ODPGRAPHIC{
\begin{tikzpicture}
 \node[owlclass] (ACTION) {
 \begin{owlclass}{Action}
  \item $(\geq 1\ \emph{executesTask}.\texttt{Task})$
 \end{owlclass}
 };
 \node[owlclass,below=0.6cm of ACTION] (TASK) {
 \begin{owlclass}{Task}
  \item $(\forall \emph{isExecutedIn}.\texttt{Action})$
 \end{owlclass}
 };
 \draw (ACTION) edge[thick,-,dashed,blue!60] (TASK);
\end{tikzpicture}
}
\end{ODP}

\newpage
\section{Quality -- Region ODP}
\begin{ODP}{Quality -- Region}
\ODPDESCRIPTION{This ODP allows structuring information about the properties of an Entity. It distinguishes between dependent aspects of the Entity (things that cannot exist without the Entity itself existing), and the values that may be ascribed to those aspects. These values may be points in some Region. Note that a Region may be a finite set of discrete labels, allowing for ``qualitative'' descriptions, but more often a Region is some dimensional space allowing ``quantitative'' descriptions. A Region may contain a single point, in cases where the value of a Quality is known precisely.}
\ODPINTENT{To distinguish between an aspect of an Entity and a particular numerical description of it.}
\ODPDOMAIN{
  \texttt{Measurement},
  \texttt{Object representation},
  \texttt{Environment representation},
  \texttt{Execution status}}
\ODPDEFINEDIN{DUL.owl}
\ODPQUESTION{
  \emph{What qualities does an entity have?}
  \emph{What are possible values for a quality?}
  \emph{What is the actual value of a quality for a particular entity (at a particular time)?}}
\ODPGRAPHIC{
\begin{tikzpicture}
 \node[owlclass] (QUALITY) {
 \begin{owlclass}{Quality}
  \item $(\exists \emph{isQualityOf}.\texttt{Entity})$
  \item $(\exists \emph{hasRegion}.\texttt{Region})$
 \end{owlclass}
 };
 \node[owlclass,below=0.6cm of QUALITY] (REGION) {
 \begin{owlclass}{Region}
  \item $(\exists \emph{isRegionFor}.\texttt{Quality})$
 \end{owlclass}
 };
 \draw (QUALITY) edge[thick,-,dashed,blue!60] (REGION);
\end{tikzpicture}
}
\end{ODP}

\newpage
\section{Process vs. Action ODP}
\begin{ODP}{Process vs. Action}
\ODPDESCRIPTION{An Action is an Event with at least one Agent participant, such that this Agent has a Task, often defined by a Plan or Workflow, which it executes through the Action. A Process is an Event for which no such commitments have been made. In DUL, these classes are not disjoint, allowing a particular event individual to be classified as either, depending on whether we care to record an agent and its goals or not. In EASE, we use Process as a top-level class for events with no agentive participant.}
\ODPINTENT{To represent the intentional and agentive structure-- or lack thereof-- behind Events.}
\ODPDOMAIN{
  \texttt{Event classification},
  \texttt{Event narratives}}
\ODPDEFINEDIN{DUL.owl}
\ODPQUESTION{
  \emph{Is there anyone responsible for the event?}
  \emph{What are they trying to do?}
  \emph{How did an event unfold?}}
\ODPGRAPHIC{
\begin{tikzpicture}
 \node[owlclass] (Action) {
 \begin{owlclass}{Action}
  \item \texttt{Event}
  \item $(\exists \emph{hasParticipant}.\texttt{Agent})$
 \end{owlclass}
 };
 \node[owlclass,below=0.6cm of ACTION] (PROCESS) {
 \begin{owlclass}{Process}
  \item \texttt{Event}
 \end{owlclass}
 };
\end{tikzpicture}
}
\end{ODP}

\newpage
\section{Designed Artifact ODP}
\begin{ODP}{Designed Artifact}
\ODPDESCRIPTION{A DesignedArtifact is a physical object described by a Design. In EASE, Designs refer to the form, but also the function of an object. This allows us to say that an object is ``for'' a particular purpose, even though it might be used for something else instead. For example, a cup is a BeverageContainer but can be used as a Flowerpot. Designs form a hierarchy of specificity, for example $DesignMilkContainer \sqsubseteq DesignBeverageContainer \sqsubseteq DesginContainer \sqsubseteq Design$. The justification for this pattern is that the type of an object is rigid, but the roles it plays in events change. A naive taxonomy, without a notion similar to Design, cannot tackle the fact that objects are usable in several ways beyond the obvious; a hammer isn't always a hammer, sometimes it's a paperweight. On the other hand, a usable ontology of objects must take into account how human users refer to objects by their default use.}
\ODPINTENT{To explicate the intuitive classification human users would have of objects, based on their default uses.}
\ODPDOMAIN{
  \texttt{Object classification},
  \texttt{Event narratives}}
\ODPDEFINEDIN{DUL.owl, EASE.owl, EASE-middle.owl}
\ODPQUESTION{
  \emph{What sort of object is this?}
  \emph{What is the intended use of the object?}
  \emph{How did an event unfold?}}
\ODPGRAPHIC{
\begin{tikzpicture}
 \node[owlclass] (DESIGNEDARTIFACT) {
 \begin{owlclass}{DesignedArtifact}
  \item \texttt{PhysicalArtifact}
  \item $(\exists\emph{isDescribedBy}.\texttt{Design})$
 \end{owlclass}
 };
 \node[owlclass,below=0.6cm of DESIGNEDARTIFACT] (DESIGN) {
 \begin{owlclass}{Design}
  \item \texttt{Description}
 \end{owlclass}
 };
 \draw (DESIGNEDARTIFACT) edge[thick,-,dashed,blue!60] (DESIGN);
\end{tikzpicture}
}
\end{ODP}

\newpage
\section{Transient ODP}
\begin{ODP}{Transient}
\ODPDESCRIPTION{This pattern addresses the fact that objects change by undergoing/taking part in Processes. For example, the PancakeMix becomes, through Baking, a Pancake. However, the ontological status of the object while the process takes place is unclear: the object placed on the frying pan is not a Pancake until Baking finishes, but it's not PancakeMix either once it begins to coagulate. In the EASE approach, the object in-between such characterizations is a Transient. A Transient transitionsFrom an Object, and possibly transitionsTo an Object. It may also be the case that a Transient transitionsBack to an Object, to indicate that once the process completes, the same Object is restored; this would be the case for example for catalysts in chemistry, or a loaf of bread after slicing, if there is enough bread left.}
\ODPINTENT{Ontological classification for objects undergoing type changes.}
\ODPDOMAIN{
  \texttt{Object classification},
  \texttt{Event narratives}}
\ODPDEFINEDIN{EASE.owl, EASE-middle.owl}
\ODPQUESTION{
  \emph{What sort of object is this?}
  \emph{What objects ``went'' in the making of another?}
  \emph{Does an object preserve or restore its identity after change?}}
\ODPGRAPHIC{
\begin{tikzpicture}
 \node[owlclass] (TRANSIENT) {
 \begin{owlclass}{Transient}
  \item \texttt{Object}
  \item $(\exists\emph{transitionsFrom}.\texttt{Object})$
  \item $(\forall\emph{transitionsTo}.\texttt{Object})$
 \end{owlclass}
 };
 \node[owlclass,below=0.6cm of TRANSIENT] (TRANSITIONSBACK) {
 \begin{owlclass}{transitionsBack}
  \item \texttt{transitionsFrom}
  \item \texttt{transitionsTo}
 \end{owlclass}
 };
\end{tikzpicture}
}
\end{ODP}

\newpage
\section{States}
\begin{ODP}{State, Configuration, Gestallt}
\ODPDESCRIPTION{A state is a configuration of the world that is construed to be stable on its own. Outside disturbances may cause state transitions, and the settling into some other, self-stable configuration. A State is also characterized by a Description, that indicates things such as what kind of entities participate in the state, what relations might exist between them, what regions may be used by particular qualities of the participants. This Description is, in general, referred to as a Configuration, however some common examples are Goals-- describe desired states of the world--, Norms-- describe states that should be kept--, and Diagnoses-- describe a state that causes certain observable symptoms. States are classified by Gestallts.}
\ODPINTENT{Ontological representation for situations in the world that are cognitively construed as stable arrangements of entities.}
\ODPDOMAIN{
  \texttt{Event classification},
  \texttt{Event narratives}}
\ODPDEFINEDIN{EASE-STATE.owl}
\ODPQUESTION{
  \emph{What are stable arrangements?}
  \emph{What is meant by ``state'' of the world?}
  \emph{What characterizes a state?}}
\emph{Examples}
\begin{itemize}
  \item AssemblyConnection: two objects are in a rigid connection, such that the movement of one determines the movement of the other. In this case the characterizing Configuration for this State uses several Roles-- one for each part/geometric feature belonging to the connected objects-- and puts constraints on the relative positioning of these geometric features such that they interlock to produce the rigid connection.
  \item Contact: two objects are in mechanical contact. The characterizing Configuration uses two Roles, one for each participating object, and puts constraints on the Pose qualities of the participants: the poses should be such that the participants touch.
  \item FunctionalControl: an object restricts the movement of another, at least partially. The Configuration uses the Roles Item and Restrictor. More concrete examples are Containment: the Restrictor is a Container, and the Pose quality of the Item should use the region inside the Container; and Support: both Restrictor and Item are objects, placed in such a way that the Item does not move because of gravity.
  \item PhysicallyAccessible: the Configuration for this state uses the roles Item, a Container or Protector, and optionally an Accessor and a Task, and states that an Item is either placed in a Container or protected by a Protector, but the placement of the Item and Container is such that an Accessor may nevertheless reach the Item in order to perform a Task. For a more concrete example, a DoorOpen is a kind of PhysicallyAccessible where the Protector is a door, the Item is the inside of the room behind the door, the Accessor is some person and the Task is to walk into the inside of the room.
\end{itemize}
%\ODPGRAPHIC{
%\begin{tikzpicture}
% \node[owlclass] (TRANSIENT) {
% \begin{owlclass}{Transient}
%  \item \texttt{Object}
%  \item $(\exists\emph{transitionsFrom}.\texttt{Object})$
%  \item $(\forall\emph{transitionsTo}.\texttt{Object})$
% \end{owlclass}
% };
% \node[owlclass,below=0.6cm of TRANSIENT] (TRANSITIONSBACK) {
% \begin{owlclass}{transitionsBack}
%  \item \texttt{transitionsFrom}
%  \item \texttt{transitionsTo}
% \end{owlclass}
% };
%\end{tikzpicture}
%}
\end{ODP}

% % % % % % % % % % % % % % % % % % % % % % % %
% % % % % % % % % % % % % % % % % % % % % % % %
% \section{Workflows}

% % % % % % % % % % % % % % % % % % % % % % % %
% % % % % % % % % % % % % % % % % % % % % % % %
% \section{Action Phases}

% % % % % % % % % % % % % % % % % % % % % % % %
% % % % % % % % % % % % % % % % % % % % % % % %
% \section{Transients}

