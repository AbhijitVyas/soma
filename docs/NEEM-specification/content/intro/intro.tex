\chapter{Introduction}
\chapterauthor{D. Be{\ss}ler, S. Koralewski, M. Pomarlan}

This document, referred to as the ``\neem Handbook'' hereafter,
describes the \ease system\todo{Seba: We need to define the EASE system or set at least an reference to it. In the just if someone reads this document first} for episodic memories of everyday activities.
The \neem Handbook will be updated along the progress in the CRC \ease.
It is thought to provide \ease researchers with compact but still comprehensive
information about what information is contained in \neems, and how it is represented.
\todo{DB: create overview figure showing all the components (acquisition, neem-hub, representation -- use some fancy images of the robot/vr/simulation}

\begin{figure}[h!]
\centering
\includegraphics[height=8cm]{example-image-a}
\caption{The EASE system for episodic memories}
\label{fig:architecture}
\end{figure}

\paragraph{Narrative Enabled Episodic Memories}
% From KnowRob 2.0
When somebody talks about the deciding goal in the last soccer world championship many of us can ``replay'' the episode in our ``mind's eye''.
Those episodic memories can be seen as abstract descriptions that allow us to recall detailed pieces of information from any experienced activity.
Having those detailed memories, we can use them to learn general knowledge or map similar memories to unknown situations, so we know how to behave in the given situation.
\todo{DB: refer to Figure~\ref{fig:architecture}}

% From KnowRob 2.0
\ease integrates episodic memories deeply into the knowledge acquisition, representation, and processing system. 
For every activity the agent performs, observes, prospects and reads about, it creates an episode and stores it in its memory.
An episode is best understood as a video recording that the agent makes of the ongoing activity.
In addition, those videos are enriched with a very detailed story about the actions, motions, their purposes, effects and the agent's sensor information during the activity.

% From KnowRob 2.0
We define the episodic memories created by our system narrative-enabled episodic memories (\neems).
A \neem consists of the \emph{\neem experience} and the \emph{\neem narrative}.
The \neem experience captures low-level data such as the agent's sensor information, e.g. images and forces, and records of poses of the agent and its detected objects.
\neem experiences are linked to \neem narratives, which are stories of the episode described symbolically.
These narratives contain information regarding the tasks, the context, intended goals, observed effects, etc.
The \neemexp and \neemnar combined are so rich of information that the agent can replay an episode to experience the seen activity anytime again.

\neems are representations of experiences acquired through experimentation, reading, observing, mental simulation, etc.
The main goal is to establish a common vocabulary used to annotate experience data across different tasks, scientific disciplines, and modalities of acquisition, and to define models for the representation of experience data.
The vocabulary is not just a set of atomic labels, but each label has a formal definition in an ontology.
These definitions are done such that a set of \emph{competency questions} about an activitiy can be answered by a knowledge base that is equipped with the ontology and a collection of \neems.

The \neem model is formally defined in form of an \owl ontology which is based on the DOLCE+DnS Ultralite (DUL) upper-level ontology~\cite{DOLCE2003}.
DUL is a carefully designed ontology that seeks to model general categories underlying human cognition without making any discipline-specific assumptions.
Our extensions of DUL mainly focus on characterizing different aspects of activities that were not considered in much detail in DUL, but are relevant for the autonomous robotics scope.
These extensions are part of an ontology that we have called
\soma~\footnote{\url{https://ease-crc.github.io/soma}}.
A \neem is made of several patterns defined either in \dul or in \soma.

While it is possible to create the representations listed in this document through a custom exporter, it is not advised to do so.
Instead, it is advised to interface with the
\knowrob knowledge base~\footnote{\url{https://github.com/knowrob/knowrob}}.
\knowrob provides an interface based on predicate logics that allows to interact with \neems.
The language is a collection of predicates that can be called by users to ask certain types of competency questions covering different aspects of activitiy, or to add labels and relationships in the \neemnar.
We will provide example expressions in this document that highlight how the knowledge base can be used to interact with \neems.

% % % % % % % % % % % % % % % % %
% % % % % % % % % % % % % % % % %
\section{Notation} % Domain of Discourse
\label{sec:notation}

Because in subsequent sections we will give more formal defitions for various concepts, we now review some basic notions about the main formalism we use in the ontology, as well as introduce the notation we will employ.

An ontology is a collection of logical axioms in some formal language. In the case of many ontologies available in the semantic web, as well as in the case of SOMA, this formal language is Description Logic, also known as DL. The entities that DL axioms describe can be either ``concepts'', sometimes known as ``classes'', and ``individuals''. An individual is an instance of one or more concepts. A concept may be subsumed by another. Between individuals there may be relations called ``object properties''. As a syntactic convenience, an individual can also have ``data properties'' that link it to some alphanumeric data item which is itself not considered an individual in the ontology. As an example, we might have that Alice and Bob are two individuals belonging to the concept \concept{Human}, and that the object property \relation{hasChild} connects Alice to Bob, i.e. the relation asserts that Alice has Bob as a child. We may also have that Alice \relation{hasHeight} ``1.7m'', where ``1.7m'' is not an individual in the ontology but rather a data item.

Accordingly, there are axioms that describe concepts or object/data properties, and axioms that describe individuals and the relations between them. The set of concepts, object and data properties, and the axioms that define them is called the Tbox, or terminological part, and we will denote it by $\mathcal{T}$. The set of individuals and the axioms describing them is called the Abox, or assertion part, and we will denote it by $\mathcal{A}$.

It is useful when describing concepts to emphasize the concept names such that it is clear we reference the concept, and not the colloquial word. As such, \concept{Concepts} and \relation{relations} will be written in a different font. Note that the name of a concept always starts with an uppercase letter, whereas the name of a relation with a lowercase one. Any word appearing in a concept or relation name after the first one will always begin with an uppercase letter.

Ontologies are meant to build on one another, and it is not uncommon for an ontology to collect thousands of concepts from external ontologies it imports. To prevent name clashes, in actual usage the names of concepts, relations, and individuals are often name-spaced. In this document, since we mostly talk about concepts from the SOMA ontologies, the namespace will not be made explicit. An exception will be made in some diagrams where we reference concepts defined in more basic ontologies, such as those used to define the Ontology Web Language (OWL). An example is a name such as ``xsd:double''; in this case, ``xsd'' is the namespace.

% % % % % % % % % % % % % % % % %
% % % % % % % % % % % % % % % % %
\section{Scope} % Domain of Discourse
\label{sec:scope}
%\todo{Mihai: write about competency questions here. Probably would be good to provide a list of them that we intend to cover with the information contained in neems}

Our work aims to provide knowledge modeling for robotic manipulation and autonomous robot control. \todo{DB: this leaves out the neem-hub/acquistion/data aspects}
Such knowledge must include aspects of interaction forces and motion characteristics of objects participating in an action, since it is these physical and geometric considerations that are crucial to determining whether, or to what degree, an action is successful. Ideally, the knowledge we formalize in this collection of ontologies will allow a single, general control program to adapt itself, via reasoning, and generate adequate behavior in a large variety of settings, for different tasks and participating objects.
\todo{DB: I think this paragraph is focussed too much on the knowledge aspect. We should mention machine learning models that are to be trained from neems, collections of neems stored on the neem hub etc. }
% NOTE: taken from SOMA paper
%The broad scope of our work is everyday object manipulation tasks in autonomous robot control, and in particular the motion and force characteristics of objects that interact with each other.
%The research question driving us is whether a single general control program can be written that can generate adequate behavior in many different contexts: for different tasks, objects, and environments.
%\todo{somebody needs to rephrase this! it is copied from SOMA paper}

The kinds of knowledge a robot needs for competent performance of its tasks are varied. Usually, knowledge modelling in robotics and AI has focused on a symbolic level, of actions treated as black boxes that relate to a larger plan by means of their preconditions and effects. Actions are also very underspecified when described in spoken commands. This abstract level of description however is insufficient; the physical details of the actions matter. For example, the angle and speed with which a pitcher is moved, and the amount of liquid in it, determines whether there will be spillage. A robot needs to choose appropriate parameters for its actions, and infer these parameters when they are left unspecified in a command.
% NOTE: taken from SOMA paper
%One of the challenges is that, using such a general plan, the agent needs to fill the knowledge gaps between abstract instructions included in the plan and the realization of context specific behavior. That is, for example, the many ways of how humans perform a pouring task depending on the source from which is poured, the destination, and the substance that is to be poured.
%Another challenge is that object manipulation tasks may fail if the agent does not perform the motions competently and well. This is caused by the agent choosing inappropriate parametrization of its control-level functions.
%\todo{somebody needs to rephrase this! it is copied from SOMA paper}

Such inference requires the robotic agent to be equipped with common-sense and intuitive physics knowledge, as well as an abstract task and object model, and knowledge of how to apply these models in a given situation. SOMA is an attempt to support each of these requirements. A brief list of some of the over-arching competency questions follows.

%%\begin{itemize}
%%    \item \emph{How are actions conceptualized?} What is an Action, how does it relate to other concepts an Agent might have about the world? What is the purpose of an Action?
%%    \item \emph{How are possibilities for action formalized?} How does an agent model what is made available by the environment? How does an agent model what it is itself capable of?
%%    \item \emph{What is the structure of an Action?} How do several actions make up another? What objects participate in an action and with what roles?
%%    \item \emph{How are qualitative and quantitative features of the world represented?} What is the parameter set of an action? What regions can values for these parameters occupy? What is a good parameterization and how can one be found?
%%\end{itemize}

\begin{itemize}
    \item \emph{How are actions conceptualized?} What is an action, how does it relate to other concepts an agent might have about the world? What is the purpose of an action?
    \item \emph{What is the structure of an Action?} How do several actions make up another? What objects participate in an action and with what roles?
    \item \emph{How are qualitative and quantitative features of the world represented?} What is the parameter set of an action? What regions can values for these parameters occupy? What is a good parameterization and how can one be found?
    \item \emph{How are objects conceptualized?} What roles can an object play? What actions can it take part in? What kinds of objects are necessary for an action?
    \item \emph{How is an Action recorded and described?} What is the relevant data to capture how an action unfolded? What are the relevant pieces of contextual information for describing an action that has actually occurred? What was the outcome of the action, in particular, to what extent did it match the goal?
\end{itemize}
\todo{DB: could we add some motion/force questions here?}
\todo{DB: could we add some questions that require a ML model learned from neems?}

% NOTE: taken from SOMA paper
%The employment of a general plan thus requires an abstract task and object model, and a mechanism to apply this abstract knowledge in situational context.
%To achieve this, an agent needs to be equipped with the necessary common-sense and intuitive physics knowledge, which is what SOMA\todo{The acronym is no where defined} attempts.
%\todo{somebody needs to rephrase this! it is copied from SOMA paper}

% % % % % % % % % % % % % % % % %
% % % % % % % % % % % % % % % % %
\section{Overview}
\label{sec:overview}

\lipsum[4]
\todo{DB: Michael has asked for an overview section. Probaby would be good to describe a kind-of system-view on neems here}

\begin{itemize}
 \item refer to Figure~\ref{fig:architecture}
 \item paragraph about the intention behind this document, that is that EASE is interdisciplinary, and that experience data is acquired with different modalities (robot, vr, ...), and that we establish a common representation within the CRC, and this document is supposed to provide the information for ease reasearchers needed to acquire and maintain neems
 \item paragraph about generalizability of robot behaviour (add reference to Pratt paper ``
Is a Cambrian Explosion Coming for Robotics?'')
 \item paragraph about representation (neem background, narrative, experience)
 \item paragraph about acquisition (robot, vr) -- what do they have in common? what is different?
 \item paragraph about neem hub -- purpose, potential, ... -- add some buzzwords from data science
\end{itemize}

% % % % % % % % % % % % % % % % %
% % % % % % % % % % % % % % % % %
\section{Outline} % Domain of Discourse
The purpose of this document is to provide an overview about version \neemversion of \neems.
This is, first of all, how \neems are represented using ontological categories and relationships, and time series data.
The purpose of this document is not to provide a full overview about the ontological modelling underlying \neems, but rather to concentrate on aspects that are directly relevant when a \neem is created.
These are the representations that are used in the \neemnar (Section~\ref{ch:narrative}) and \neemexp (Section~\ref{ch:experience}).
In addition, \neems are situated in an environment, and acquired by some agent executing a task by interacting with its environment (the \neembak).
However, the agent and the environment may be involved in many different \neems such that their representation is seperated from the representation of the \neemnar and \neemexp (Section~\ref{ch:background}).
Second, this document provides an overview about software tools that were developed to support the acquisition of \neems in different modalities such as a tool that can be hooked into a robot control system to auto-generate \neems (Section~\ref{ch:acquisition}).
Finally, the document provides information about the \neemhub, which is an infrastructure software for the management and curation of \neems (Section~\ref{ch:neemhub}).


