\chapter{Introduction}
\chapterauthor{D. Be{\ss}ler, S. Koralewski}

This document, referred to as the ``\neem Handbook'' hereafter,
describes the \ease system for episodic memories of everyday activities.
The \neem Handbook will be updated along the progress in the CRC \ease.
It is thought to provide \ease researchers with compact but still comprehensive
information about what information is contained in \neems, and how it is represented.

\paragraph{Narrative Enabled Episodic Memories}
\todo{Seba:Rephrase}
% From KnowRob 2.0
When somebody talks about the deciding goal in the
last soccer world championship many of us can ``replay'' the episode in our ``mind's eye''.
The memory mechanism that allows us to recall these very detailed pieces of
information from abstract descriptions is our episodic memory.
Episodic memory is powerful because it allows us to remember special
experiences we had. It can also serve as a ``repository'' from which we learn general knowledge.
\todo{somebody needs to rephrase this! it is copied from ICRA paper}

% From KnowRob 2.0
\ease integrates episodic memories deeply into the knowledge acquisition, representation, and processing
system. Whenever an agent performs, observes, prospects, and
reads about an activity, it creates an episodic memory. An episodic
memory is best understood as a video that the agent makes of the
ongoing activity coupled with a very detailed story about the actions, motions, their purposes, effects,
the behavior they generate, the images that are captured, etc.
\todo{somebody needs to rephrase this! it is copied from ICRA paper}

% From KnowRob 2.0
We term the episodic memories created by our system narrative-enabled episodic memories (\neems).
A \neem consists of the \emph{\neem experience} and the \emph{\neem narrative}.
The \neem experience is a detailed, low-level, time-indexed recording
of a certain episode. The experience contains records of poses, percepts, control signals, etc.
These can be used to replay an episode in detail.
\neem experiences are linked to \neem narratives, which are stories
that provide more abstract, symbolic descriptions of what is happening in an episode.
These narratives contain information regarding the tasks, the context, intended goals, observed effects, etc.
\todo{somebody needs to rephrase this! it is copied from ICRA paper}

\neems are representations of experiences acquired through experimentation, reading, observing, mental simulation, etc.
The main goal is to establish a common vocabulary used to annotate experience data across different tasks, scientific disciplines, and modalities of acquisition, and to define models for the representation of experience data.
The vocabulary is not just a set of atomic labels, but each label has a formal definition in an ontology.
These definitions are done such that a set of \emph{competency questions} about an activitiy can be answered by a knowledge base that is equipped with the ontology and a collection of \neems.

The \neem model is formally defined in form of an \owl ontology which is based on the DOLCE+DnS Ultralite (DUL) upper-level ontology~\cite{DOLCE2003}.
DUL is a carefully designed ontology that seeks to model general categories underlying human cognition without making any discipline-specific assumptions.
Our extensions of DUL mainly focus on characterizing different aspects of activities that were not considered in much detail in DUL, but are relevant for the autonomous robotics scope.
These extensions are part of an ontology that we have called
\soma~\footnote{\url{https://ease-crc.github.io/soma}}.
A \neem is made of several patterns defined either in \dul or in \soma.

While it is possible to create the representations listed in this document through a custom exporter, it is not advised to do so.
Instead, it is advised to interface with the
\knowrob knowledge base~\footnote{\url{https://github.com/knowrob/knowrob}}.
\knowrob provides an interface based on predicate logics that allows to interact with \neems.
The language is a collection of predicates that can be called by users to ask certain types of competency questions covering different aspects of activitiy, or to add labels and relationships in the \neemnar.
We will provide example expressions in this document that highlight how the knowledge base can be used to interact with \neems.

% % % % % % % % % % % % % % % % %
% % % % % % % % % % % % % % % % %
\section{Scope} % Domain of Discourse
\label{sec:scope}
\todo{Mihai: write about competency questions here. Probably would be good to provide a list of them that we intend to cover with the information contained in neems}

% NOTE: taken from SOMA paper
The broad scope of our work is everyday object manipulation tasks in autonomous robot control, and in particular the motion and force characteristics of objects that interact with each other.
The research question driving us is whether a single general control program can be written that can generate adequate behavior in many different contexts: for different tasks, objects, and environments.
\todo{somebody needs to rephrase this! it is copied from SOMA paper}

% NOTE: taken from SOMA paper
One of the challenges is that, using such a general plan, the agent needs to fill the knowledge gaps between abstract instructions included in the plan and the realization of context specific behavior. That is, for example, the many ways of how humans perform a pouring task depending on the source from which is poured, the destination, and the substance that is to be poured.
Another challenge is that object manipulation tasks may fail if the agent does not perform the motions competently and well. This is caused by the agent choosing inappropriate parametrization of its control-level functions.
\todo{somebody needs to rephrase this! it is copied from SOMA paper}

% NOTE: taken from SOMA paper
The employment of a general plan thus requires an abstract task and object model, and a mechanism to apply this abstract knowledge in situational context.
To achieve this, an agent needs to be equipped with the necessary common-sense and intuitive physics knowledge, which is what SOMA\todo{The acronym is no where defined} attempts.
\todo{somebody needs to rephrase this! it is copied from SOMA paper}

% % % % % % % % % % % % % % % % %
% % % % % % % % % % % % % % % % %
\section{Outline} % Domain of Discourse
The purpose of this document is to provide an overview about version \neemversion of \neems.
This is, first of all, how \neems are represented using ontological categories and relationships, and time series data.
The purpose of this document is not to provide a full overview about the ontological modelling underlying \neems, but rather to concentrate on aspects that are directly relevant when a \neem is created.
These are the representations that are used in the \neemnar (Section~\ref{ch:narrative}) and \neemexp (Section~\ref{ch:experience}).
In addition, \neems are situated in an environment, and acquired by some agent executing a task by interacting with its environment (the \neembak).
However, the agent and the environment may be involved in many different \neems such that their representation is seperated from the representation of the \neemnar and \neemexp (Section~\ref{ch:background}).
Second, this document provides an overview about software tools that were developed to support the acquisition of \neems in different modalities such as a tool that can be hooked into a robot control system to auto-generate \neems (Section~\ref{ch:acquisition}).
Finally, the document provides information about the \neemhub, which is an infrastructure software for the management and curation of \neems (Section~\ref{ch:neemhub}).


