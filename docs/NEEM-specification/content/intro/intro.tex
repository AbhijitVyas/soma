\chapter{Introduction}

This document, referred to as the ``\neem Specification'' hereafter,
describes the \ease system for episodic memories of everyday activities:
what they are, how they are represented and acquired.
% and what reasoning and learning tasks can be performed with them.
% This document requires at least rudimentary understanding of
% description logic and knowledge modeling.
The \neem Specification will be updated along the progress in the CRC \ease.
It is thought to provide \ease researchers with compact but still comprehensive
information about what information is contained in \neems, and how it is represented.

\paragraph{Narrative Enabled Episodic Memories}
% From KnowRob 2.0
When somebody talks about the deciding goal in the
last soccer world championship many of us can ``replay'' the episode in our ``mind's eye''.
The memory mechanism that allows us to recall these very detailed pieces of
information from abstract descriptions is our episodic memory.
Episodic memory is powerful because it allows us to remember special
experiences we had. It can also serve as a ``repository'' from which we learn general knowledge.

% From KnowRob 2.0
\ease integrates episodic memories deeply into the knowledge acquisition, representation, and processing
system. Whenever an agent performs, observes, prospects, and
reads about an activity, it creates an episodic memory. An episodic
memory is best understood as a video that the agent makes of the
ongoing activity coupled with a very detailed story about the actions, motions, their purposes, effects,
the behavior they generate, the images that are captured, etc.

% From KnowRob 2.0
We term the episodic memories created by our system narrative-enabled episodic memories (\neems).
A \neem consists of the \emph{\neem experience} and the \emph{\neem narrative}.
The \neem experience is a detailed, low-level, time-indexed recording
of a certain episode. The experience contains records of poses, percepts, control signals, etc.
These can be used to replay an episode in detail.
\neem experiences are linked to \neem narratives, which are stories
that provide more abstract, symbolic descriptions of what is happening in an episode.
These narratives contain information regarding the tasks, the context, intended goals, observed effects, etc.

\paragraph{Structure of this Document}
First, this document provides an overview about version \neemversion of the \neem Specification,
and outlines how we, the \ease researchers, envision the evolution of the \neem Specification
throughout the CRC \ease.
The next two sections are dedicated to the two distinct parts of \neems: experience and narrative.
For both, we describe data format in detail, providing
a compact tabular view on data structures and meaning.
Finally, we provide a concrete example for a table setting activity,
a highly relevant example for research in \ease.


\todo{Are we providing a definition of term episode?}
