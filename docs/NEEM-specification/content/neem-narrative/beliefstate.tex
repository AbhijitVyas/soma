
In EASE, we investigate everyday manipulation activities.
These involve the interaction with physical objects such as
grasping an object and putting it somewhere,
or taking a tool and applying it on an object to achieve a certain effect.
To tell a comprehensive story about its activity,
an agent needs to memorize the relation of action to
physical objects that are salient during the action.

One requirement for this is that beliefs of the agent about the existence
of physical objects must be maintained by the system.
In some cases, when the existence is known in advance,
it is sufficient to supply this knowledge to the agent through
a static ontology holding facts about existence of physical objects
in the environment of the agent.
This is, for example, useful to supply knowledge about a static
environment such as a kitchen with fixed set of appliances.
In more dynamic set-ups, however, the beliefs about the existence
of physical objects must be formed through specialized perception
methods, or through logical reasoning.
This is, for example, the case when the particular set of objects
contained in a drawer are unknown in advance.
We consider this type of information as part of the NEEM narrative
to keep these beliefs persistent, and to allow referring to
the perceived objects in action descriptions.

Episodic memories capture information about temporal situations
during which the beliefs of the agent evolve according to
perception, action, and logical inference.
NEEMs capture this evolution of beliefs.
As such, revised or false beliefs are not deleted but kept
persistently as part of the NEEM narrative.
This is to allow for deeper analysis of the agent performance,
and to provide richer input for learning algorithms.
This is realized through a 4D ontology that we describe at
the end of this section.

%%%%%%%%%%%%%%%%%%%%%%%%%%%%%
%%%%%%%%%%%%%%%%%%%%%%%%%%%%%
%%%%%%%%%%%%%%%%%%%%%%%%%%%%%
%%%%%%%%%%%%%%%%%%%%%%%%%%%%%
\subsection{Tangible Objects}
To refer to objects in action descriptions only the name of the 
object must be known to the NEEM acquisition system.
\knowrob comes with a rather comprehensive object type system,
with about XXX different object classes~\cite{knowrob-ontology}.
The type system is represented as part of the TBOX of the knowledge base.
Perceived objects are represented as instance of one of the object types
provided by the \knowrob system.
This type of information is part of the ABOX of the knowledge base.
The name of an object is usually the name of the class followed
by a underscore character and a 8-digit hash,
but could potentially be any unique name string.

\todo{introduce base class of objects, give some details about taxonomy}

In systems relying on sensor information and statistical models one has to deal
with the uncertainty coming from the use of these information sources.
This also affects beliefs about the identity of objects.
The problem of identity resolution can trivially be approached
by computing euclidean distance to known objects.
If the distance is below a certain threshold, it can often be assumed that it is the same object.
Background knowledge can be used to find better estimates for object identity:
An object is inside the gripper as long as it is grasped,
pulled down by gravity to the plane below when the grasp is released again,
and so forth.
Also, the identity of objects is often not so important for decision making,
but rather the properties of the object.
For successfully preparing a slice of bread, for example, any butter will do.
But if there are two butter packages and one is not yet opened one would rather
use the opened one to avoid it getting rancid.
When acquiring NEEMs one has to deal with this identity resolution problem
to allow for referring to objects in the narrative part of the episodic memory.

In the remainder of this section we describe the fundamental binary predicates to
describe physical objects in the belief state.

\begin{description}
\item[\textbf{rdf:type}] 
The most fundamental assertion about an object next to its name is its type.
Types are asserted through the \emph{rdf:type}
property with one of the object types as value.
Ontologies are subsumption hierarchies that derive 
knowledge about more specific concepts from more general concepts.
This allows, for example, to describe the relation between object and action
on a general level that applies to a larger class of objects and actions.
It is further possible to assert multiple types from different branches
of the type hierarchy for a single object.
For example, some mobile phone may also be used as projector while other mobile phones,
without the projector hardware, should not be classified as projector.
%%%%%%%%%%%%%%%%%%%%%%%%%%%%%
%%%%%%%%%%%%%%%%%%%%%%%%%%%%%
\item[\textbf{physicalParts}]
$Whole\ \emph{physicalParts}\ Part$ means
that $Part$ is tangible and one of the distinct parts of the
more complex tangible object $Whole$.
Manipulation is often about putting together parts to form a bigger whole.
During table setting, for example, objects are arranged such that they form
place settings for the different participants.
The objects are placed intentionally such that humans can easily reach them from their seat,
see which objects belong to which place setting, and infer what objects belong to whom.
Another example are assembly tasks during which an agent tries
to put together some mechanical parts using screwing connections, snap-in connection,
and so on to create an assembled product from scattered pieces available.
NEEMs can also represent this partonomy information through dedicated predicates
linking parts to the bigger whole.
In particular, the \ease ontology defines the following sub-properties of \emph{physicalParts}:
\emph{dangerousParts} (e.g., the blade of a knife),
\emph{electricalParts} (i.e., parts that need electrical current to work),
\emph{mealComponents} (i.e., the ingredients of a meal).
%%%%%%%%%%%%%%%%%%%%%%%%%%%%%
%%%%%%%%%%%%%%%%%%%%%%%%%%%%%
\item[\textbf{frameName}] \dots
\end{description}

\todo{some other relevant predicates? shape, color, mesh, pose?}

%%%%%%%%%%%%%%%%%%%%%%%%%%%%%
%%%%%%%%%%%%%%%%%%%%%%%%%%%%%
%%%%%%%%%%%%%%%%%%%%%%%%%%%%%
%%%%%%%%%%%%%%%%%%%%%%%%%%%%%
\subsection{Kinematic Objects}
\begin{enumerate}
 \item \todo{object parts can be connected through joints}
 \item \todo{what types of joints exist?}
 \item \todo{what predicates can be used to describe them?}
\end{enumerate}

%%%%%%%%%%%%%%%%%%%%%%%%%%%%%
%%%%%%%%%%%%%%%%%%%%%%%%%%%%%
%%%%%%%%%%%%%%%%%%%%%%%%%%%%%
%%%%%%%%%%%%%%%%%%%%%%%%%%%%%
\subsection{Environment Maps}
Semantic maps are detailed representations of the environment of some agent.
These include geometrical and visual information about objects and appliances,
their functional decomposition, and their state.
Modern game engines reach an impressive level of detail, with individual
leaves falling down trees, etc.
In EASE, we try to reach this level of detail in our semantic map representations.
We do this to provide very comprehensive information about situated experiences
from which general knowledge can be learned.
We believe that this high level of detail will allow us to learn more robust
models and with less training data then would be possible with
a more abstracted representation of environments.

The \emph{SemanticMap} itself could be a \emph{Room}, a \emph{Building}, or some other type of connected
region in which some agent can do navigation and manipulation.
The ontology defines some more specific room classes including
\emph{Kitchen}, \emph{LivingRoom} and \emph{StoreRoom}.
This is useful, for example, to correlate objects with their likely storage room,
or to relate the rooms with common activities performed in them.

\begin{description}
%%%%%%%%%%%%%%%%%%%%%%%%%%%%%
%%%%%%%%%%%%%%%%%%%%%%%%%%%%%
\item[\textbf{describedInMap}]
$Object\ \emph{describedInMap}\ Map$ means \dots
\end{description}

\todo{also describe how rooms relate to buildings, etc.?}
\todo{something about the state?}


\subsection{Temporal Representation}
\dots