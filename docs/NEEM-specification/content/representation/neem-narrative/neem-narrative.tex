
% % % % % % % % % % % % % % % % % % % % % % % %
% % % Prelude
% % % % % % % % % % % % % % % % % % % % % % % %
\newcommand{\givenODPNAME}{}
\newcommand{\givenODPINTENT}{}
\newcommand{\givenODPDEFINEDIN}{}
\newcommand{\givenODPDESCRIPTION}{}
\newcommand{\givenODPGRAPHIC}{}
\newcommand{\givenODPDOMAIN}{}
\newcommand{\givenODPQUESTION}{}
\newcommand{\ODPINTENT}[1]     {\renewcommand{\givenODPINTENT}{#1}}
\newcommand{\ODPDEFINEDIN}[1]  {\renewcommand{\givenODPDEFINEDIN}{#1}}
\newcommand{\ODPDESCRIPTION}[1]{\renewcommand{\givenODPDESCRIPTION}{#1}}
\newcommand{\ODPGRAPHIC}[1]    {\renewcommand{\givenODPGRAPHIC}{#1}}
\newcommand{\ODPDOMAIN}[1]     {\renewcommand{\givenODPDOMAIN}{#1}}
\newcommand{\ODPQUESTION}[1]   {\renewcommand{\givenODPQUESTION}{#1}}
\newcommand{\OPDinit}{
  \renewcommand{\givenODPINTENT}{REQUIRED!}
  \renewcommand{\givenODPDEFINEDIN}{REQUIRED!}
  \renewcommand{\givenODPDESCRIPTION}{REQUIRED!}
  \renewcommand{\givenODPGRAPHIC}{REQUIRED!}
  \renewcommand{\givenODPQUESTION}{}
  \renewcommand{\givenODPDOMAIN}{}
  \renewcommand{\labelitemi}{$\mathbf{\sqsubseteq}$}
}

\newenvironment{owlclass}[2][,] {
  \begin{minipage}{5.0cm}
  \begin{center}
  \texttt{\bf#2} \\[-0.2cm]
  \par\noindent\rule{\textwidth}{0.4pt}
  \vspace{-0.6cm}
  \begin{itemize}[#1]
  \raggedright} {
  % % % % % %
  \end{itemize}
  \end{center}
  \end{minipage}
}

\newenvironment{ODP}[1]{
\OPDinit
\renewcommand{\givenODPNAME}{#1}
}{
\givenODPDESCRIPTION
\begin{figure}[htb]
\begin{minipage}{0.45\textwidth}
\begin{tabular}{ p{1.8cm} p{3.2cm} }
\toprule
% {\it\bf Name}                 & \emph{\givenODPNAME} \\
{\it\bf Intent}               & \givenODPINTENT \\
{\it\bf Domains}              & \givenODPDOMAIN \\
{\it\bf Competency Questions} & \givenODPQUESTION \\
{\it\bf Defined in}           & \givenODPDEFINEDIN \\
\bottomrule
\end{tabular}
\end{minipage}
\begin{minipage}{0.55\textwidth}
\begin{center}
\givenODPGRAPHIC
\end{center}
\end{minipage}
\caption{The \emph{\givenODPNAME} ODP.}
\end{figure}
}


\tikzset{owlclass/.style={draw=blue!40,fill=blue!20,rounded corners}}

% % % % % % % % % % % % % % % % %
\section{NEEM-Narrative}
\label{ch:narrative}

% NOTE: taken from KnowRob2.0
% One of the most powerful components of the human memory system is the
% episodic memory. 
When somebody talks about the deciding goal in the
last soccer world championship many of us can ``replay'' the episode
in our ``mind's eye''. %, or if somebody talks about a cross court hit
%in tennis we can recall how such a hit feels in the arm.
The memory
mechanism that allows us to recall these very detailed pieces of
information from abstract descriptions is our episodic memory.
Episodic memory is powerful because it allows us to remember special
experiences we had. It can also serve as a ``repository'' from which
we learn general knowledge.

% NOTE: taken from KnowRob2.0
EASE integrates episodic memories
deeply into the knowledge acquisition, representation, and processing
system. Whenever a robotic agent performs, observes, prospects, and
reads about an activity, it creates an episodic memory. An episodic
memory is best understood as a video that the agent makes of the
ongoing activity 
%in its inner world knowledge base
coupled with a very
detailed story about the actions, motions, their purposes, effects,
the behavior they generate, the images that are captured, etc.

% % % % % % % % % % % % % % % % %
% % % % % % % % % % % % % % % % %
\subsection{Scope} % Domain of Discourse
\label{sec:narrative:scope}

% NOTE: taken from SOMA paper
The broad scope of our work is everyday object manipulation tasks in autonomous robot control, and in particular the motion and force characteristics of objects that interact with each other.
The research question driving us is whether a single general control program can be written that can generate adequate behavior in many different contexts: for different tasks, objects, and environments.

% NOTE: taken from SOMA paper
One of the challenges is that, using such a general plan, the agent needs to fill the knowledge gaps between abstract instructions included in the plan and the realization of context specific behavior. That is, for example, the many ways of how humans perform a pouring task depending on the source from which is poured, the destination, and the substance that is to be poured.
Another challenge is that object manipulation tasks may fail if the agent does not perform the motions competently and well. This is caused by the agent choosing inappropriate parametrization of its control-level functions.

% NOTE: taken from SOMA paper
The employment of a general plan thus requires an abstract task and object model, and a mechanism to apply this abstract knowledge in situational context.
To achieve this, an agent needs to be equipped with the necessary common-sense and intuitive physics knowledge, which is what SOMA attempts.

\textbf{TODO: write about competency questions}

% % % % % % % % % % % % % % % % %
% % % % % % % % % % % % % % % % %
\subsection{Upper-Level Ontology}
\label{sec:narrative:commitments}

% NOTE: taken from SOMA paper
We decided to base our model on the DOLCE+DnS Ultralite (DUL) foundational framework~\cite{DOLCE2003}.
This decision is greatly motivated by their underlying ontological commitments.
%The decision to base our model on the DOLCE+DnS Ultralite foundational framework, is greatly motivated by their underlying ontological commitments.
Firstly, DUL is not a revisionary model, but seeks to express stands that shape human cognition. Furthermore it assumes a reductionist approach -- rather than capturing, for example, the flexibility of our usage of objects via multiple inheritance in a multiplicative manner, we commit to a reduced {\it ground} classification and use a {\it descriptive} approach for handling this flexibility. For this a primary branch of the ontology represents the ground {\bf physical model}, e.g. objects and actions, while a secondary branch represents the {\bf social model}, e.g. roles and tasks. All entities in the social branch would not exist without humans, i.e. they constitute social objects that represent concepts about or descriptions of ground elements. 

% NOTE: taken from SOMA paper
Every axiomatization in the physical branch can, therefore, be regarded as expressing some physical context whereas axiomatizations in the descriptive social branch are used to express social contexts. A set of dedicated relations is provided that connect both branches. For example, as detailed in Section~\ref{subsec:roles}, the relation \emph{classifies} connects ground objects, e.g. a hammer, with the roles they can play, i.e. potential classifications. Thus, we can state that a hammer can in some context be conceptualized as a murder weapon, a paper weight or a door stopper. Nevertheless, neither its ground ontological classification as a tool will change nor will hammers be subsumed as kinds of door stoppers, paper weights or weapons via multiple inheritance. Following a quick overview of the central modules of SOMA where these commitments apply, we will provide detailed examples of where and how our commitments apply in Sections~\ref{sec:perdurants} and~\ref{sec:endurants}.

% % % % % % % % % % % % % % % % %
% % % % % % % % % % % % % % % % %
\subsection{SOMA Taxonomy}
\label{sec:narrative:taxonomy}

\textbf{TODO: provide a broad overview about types in SOMA, what branches the ontology has. Show a Figure, maybe use this Figure to indicate in subsections at which branch in the taxonomy we are?}

% % % % % % % % % % % % % % % % %
% % % % % % % % % % % % % % % % %
\subsection{Event Branch in SOMA}
\begin{itemize}
\item show SOMA Figure highlight event branch
\item write about different types of events
\item write about relationships between events
\item write about relationships between events and objects
\item provide examples of everything
\item display examples as graph figures
\end{itemize}

\subsubsection{Occurs}
\textbf{events occur, they have a time interval}

\subsubsection{Participation}
\textbf{objects participate in events}

\subsubsection{Classification}

\paragraph{Process vs. Action ODP}
\begin{ODP}{Process vs. Action}
\ODPDESCRIPTION{An Action is an Event with at least one Agent participant, such that this Agent has a Task, often defined by a Plan or Workflow, which it executes through the Action. A Process is an Event for which no such commitments have been made. In DUL, these classes are not disjoint, allowing a particular event individual to be classified as either, depending on whether we care to record an agent and its goals or not. In EASE, we use Process as a top-level class for events with no agentive participant.}
\ODPINTENT{To represent the intentional and agentive structure-- or lack thereof-- behind Events.}
\ODPDOMAIN{
  \texttt{Event classification},
  \texttt{Event narratives}}
\ODPDEFINEDIN{DUL.owl}
\ODPQUESTION{
  \emph{Is there anyone responsible for the event?}
  \emph{What are they trying to do?}
  \emph{How did an event unfold?}}
\ODPGRAPHIC{
\begin{tikzpicture}
 \node[owlclass] (ACTION) {
 \begin{owlclass}{Action}
  \item \texttt{Event}
  \item $(\exists \emph{hasParticipant}.\texttt{Agent})$
 \end{owlclass}
 };
 \node[owlclass,below=0.6cm of ACTION] (PROCESS) {
 \begin{owlclass}{Process}
  \item \texttt{Event}
 \end{owlclass}
 };
\end{tikzpicture}
}
\end{ODP}

\textbf{tasks are used to classify events}
\paragraph{Task Execution ODP}
\begin{ODP}{Task Execution}
\ODPDESCRIPTION{
This ODP allows to make assertions on roles played by agents
without involving the agents that play that roles, and vice versa.
It allows to express neither the context type in which tasks are defined,
not the particular context in which the action is carried out.
Moreover, it does not allow to express the time at which
the task is executed through the action
(for actions that do not solely execute that certain task).}
\ODPINTENT{To represent actions through which tasks are executed. }
\ODPDOMAIN{
  \texttt{Organization},
  \texttt{Management},
  \texttt{Scheduling},
  \texttt{Workflow}}
\ODPDEFINEDIN{DUL.owl}
\ODPQUESTION{
  \emph{Which task is executed through this action?}
  \emph{What actions can execute that task?}}
\ODPGRAPHIC{
\begin{tikzpicture}
 \node[owlclass] (ACTION) {
 \begin{owlclass}{Action}
  \item $(\geq 1\ \emph{executesTask}.\texttt{Task})$
 \end{owlclass}
 };
 \node[owlclass,below=0.6cm of ACTION] (TASK) {
 \begin{owlclass}{Task}
  \item $(\forall \emph{isExecutedIn}.\texttt{Action})$
 \end{owlclass}
 };
 \draw (ACTION) edge[thick,-,dashed,blue!60] (TASK);
\end{tikzpicture}
}
\end{ODP}

\subsubsection{Composition / Constituency}
\textbf{events may have sub-events, or some other events that are constituents of it. the structure of events is described in plans. Plans use Allen's relations to structure the event, etc.}

\paragraph{State, Configuration, Gestallt ODP}
\begin{ODP}{State, Configuration, Gestallt}
\ODPDESCRIPTION{A state is a configuration of the world that is construed to be stable on its own. Outside disturbances may cause state transitions, and the settling into some other, self-stable configuration. A State is also characterized by a Description, that indicates things such as what kind of entities participate in the state, what relations might exist between them, what regions may be used by particular qualities of the participants. This Description is, in general, referred to as a Configuration, however some common examples are Goals-- describe desired states of the world--, Norms-- describe states that should be kept--, and Diagnoses-- describe a state that causes certain observable symptoms. States are classified by Gestallts. \textbf{TODO}: Generalize this and/or split it: classification by an event type and structuring by a description are different ODPs in the current list. All three event subclasses (Actions, Processes, States) now are each part of their own Event-Concept-Description triad, where the Concept classifies the Event, and the Description, by describing the classifying Concept, structures the Event.
}
\ODPINTENT{Ontological representation for situations in the world that are cognitively construed as stable arrangements of entities.}
\ODPDOMAIN{
  \texttt{Event classification},
  \texttt{Event narratives}}
\ODPDEFINEDIN{EASE-STATE.owl}
\ODPQUESTION{
  \emph{What are stable arrangements?}
  \emph{What is meant by ``state'' of the world?}
  \emph{What characterizes a state?}}
\emph{Examples}
%\begin{itemize}
%  \item AssemblyConnection: two objects are in a rigid connection, such that the movement of one determines the movement of the other. In this case the characterizing Configuration for this State uses several Roles-- one for each part/geometric feature belonging to the connected objects-- and puts constraints on the relative positioning of these geometric features such that they interlock to produce the rigid connection.
%  \item Contact: two objects are in mechanical contact. The characterizing Configuration uses two Roles, one for each participating object, and puts constraints on the Pose qualities of the participants: the poses should be such that the participants touch.
%  \item FunctionalControl: an object restricts the movement of another, at least partially. The Configuration uses the Roles Item and Restrictor. More concrete examples are Containment: the Restrictor is a Container, and the Pose quality of the Item should use the region inside the Container; and Support: both Restrictor and Item are objects, placed in such a way that the Item does not move because of gravity.
%  \item PhysicallyAccessible: the Configuration for this state uses the roles Item, a Container or Protector, and optionally an Accessor and a Task, and states that an Item is either placed in a Container or protected by a Protector, but the placement of the Item and Container is such that an Accessor may nevertheless reach the Item in order to perform a Task. For a more concrete example, a DoorOpen is a kind of PhysicallyAccessible where the Protector is a door, the Item is the inside of the room behind the door, the Accessor is some person and the Task is to walk into the inside of the room.
%\end{itemize}
\end{ODP}

\subsubsection{Interpretation}
\textbf{events are interpreted through situations that satisfy some description of the task that is executed.}

% % % % % % % % % % % % % % % % %
% % % % % % % % % % % % % % % % %
\subsection{Object Branch in SOMA}

\subsubsection{Classification}

%\newpage
\paragraph{Designed Artifact ODP}
\begin{ODP}{Designed Artifact}
\ODPDESCRIPTION{A DesignedArtifact is a physical object described by a Design. In EASE, Designs refer to the form, but also the function of an object. This allows us to say that an object is ``for'' a particular purpose, even though it might be used for something else instead. For example, a cup is a BeverageContainer but can be used as a Flowerpot. Designs form a hierarchy of specificity, for example $DesignMilkContainer \sqsubseteq DesignBeverageContainer \sqsubseteq DesginContainer \sqsubseteq Design$. The justification for this pattern is that the type of an object is rigid, but the roles it plays in events change. A naive taxonomy, without a notion similar to Design, cannot tackle the fact that objects are usable in several ways beyond the obvious; a hammer isn't always a hammer, sometimes it's a paperweight. On the other hand, a usable ontology of objects must take into account how human users refer to objects by their default use.}
\ODPINTENT{To explicate the intuitive classification human users would have of objects, based on their default uses.}
\ODPDOMAIN{
  \texttt{Object classification},
  \texttt{Event narratives}}
\ODPDEFINEDIN{DUL.owl, EASE.owl, EASE-middle.owl}
\ODPQUESTION{
  \emph{What sort of object is this?}
  \emph{What is the intended use of the object?}
  \emph{How did an event unfold?}}
\ODPGRAPHIC{
\begin{tikzpicture}
 \node[owlclass] (DESIGNEDARTIFACT) {
 \begin{owlclass}{DesignedArtifact}
  \item \texttt{PhysicalArtifact}
  \item $(\exists\emph{isDescribedBy}.\texttt{Design})$
 \end{owlclass}
 };
 \node[owlclass,below=0.6cm of DESIGNEDARTIFACT] (DESIGN) {
 \begin{owlclass}{Design}
  \item \texttt{Description}
 \end{owlclass}
 };
 \draw (DESIGNEDARTIFACT) edge[thick,-,dashed,blue!60] (DESIGN);
\end{tikzpicture}
}
\end{ODP}

\subsubsection{Qualification}

\paragraph{Quality -- Region ODP}
\begin{ODP}{Quality -- Region}
\ODPDESCRIPTION{This ODP allows structuring information about the properties of an Entity. It distinguishes between dependent aspects of the Entity (things that cannot exist without the Entity itself existing), and the values that may be ascribed to those aspects. These values may be points in some Region. Note that a Region may be a finite set of discrete labels, allowing for ``qualitative'' descriptions, but more often a Region is some dimensional space allowing ``quantitative'' descriptions. A Region may contain a single point, in cases where the value of a Quality is known precisely.}
\ODPINTENT{To distinguish between an aspect of an Entity and a particular numerical description of it.}
\ODPDOMAIN{
  \texttt{Measurement},
  \texttt{Object representation},
  \texttt{Environment representation},
  \texttt{Execution status}}
\ODPDEFINEDIN{DUL.owl}
\ODPQUESTION{
  \emph{What qualities does an entity have?}
  \emph{What are possible values for a quality?}
  \emph{What is the actual value of a quality for a particular entity (at a particular time)?}}
\ODPGRAPHIC{
\begin{tikzpicture}
 \node[owlclass] (QUALITY) {
 \begin{owlclass}{Quality}
  \item $(\exists \emph{isQualityOf}.\texttt{Entity})$
  \item $(\exists \emph{hasRegion}.\texttt{Region})$
 \end{owlclass}
 };
 \node[owlclass,below=0.6cm of QUALITY] (REGION) {
 \begin{owlclass}{Region}
  \item $(\exists \emph{isRegionFor}.\texttt{Quality})$
 \end{owlclass}
 };
 \draw (QUALITY) edge[thick,-,dashed,blue!60] (REGION);
\end{tikzpicture}
}
\end{ODP}

\subsubsection{Transformation}
\paragraph{Transient ODP}
\begin{ODP}{Transient}
\ODPDESCRIPTION{This pattern addresses the fact that objects change by undergoing/taking part in Processes. For example, the PancakeMix becomes, through Baking, a Pancake. However, the ontological status of the object while the process takes place is unclear: the object placed on the frying pan is not a Pancake until Baking finishes, but it's not PancakeMix either once it begins to coagulate. In the EASE approach, the object in-between such characterizations is a Transient. A Transient transitionsFrom an Object, and possibly transitionsTo an Object. It may also be the case that a Transient transitionsBack to an Object, to indicate that once the process completes, the same Object is restored; this would be the case for example for catalysts in chemistry, or a loaf of bread after slicing, if there is enough bread left.}
\ODPINTENT{Ontological classification for objects undergoing type changes.}
\ODPDOMAIN{
  \texttt{Object classification},
  \texttt{Event narratives}}
\ODPDEFINEDIN{EASE.owl, EASE-middle.owl}
\ODPQUESTION{
  \emph{What sort of object is this?}
  \emph{What objects ``went'' in the making of another?}
  \emph{Does an object preserve or restore its identity after change?}}
\ODPGRAPHIC{
\begin{tikzpicture}
 \node[owlclass] (TRANSIENT) {
 \begin{owlclass}{Transient}
  \item \texttt{Object}
  \item $(\exists\emph{transitionsFrom}.\texttt{Object})$
  \item $(\forall\emph{transitionsTo}.\texttt{Object})$
 \end{owlclass}
 };
 \node[owlclass,below=0.6cm of TRANSIENT] (TRANSITIONSBACK) {
 \begin{owlclass}{transitionsBack}
  \item \texttt{transitionsFrom}
  \item \texttt{transitionsTo}
 \end{owlclass}
 };
\end{tikzpicture}
}
\end{ODP}

% % % % % % % % % % % % % % % % %
% % % % % % % % % % % % % % % % %

%\subsection{Actions}

%\todo{Describe ACT, PROC, STATE from the SOMA paper}

%\subsubsection{Ontology Design Patterns (ODP)}

%\newpage
%\subsubsection{Process vs. Action ODP}

%\begin{ODP}{Process vs. Action}
%\ODPDESCRIPTION{An Action is an Event with at least one Agent participant, such that this Agent has a Task, often defined by a Plan or Workflow, which it executes through the Action. A Process is an Event for which no such commitments have been made. In DUL, these classes are not disjoint, allowing a particular event individual to be classified as either, depending on whether we care to record an agent and its goals or not. In EASE, we use Process as a top-level class for events with no agentive participant.}
%\ODPINTENT{To represent the intentional and agentive structure-- or lack thereof-- behind Events.}
%\ODPDOMAIN{
%  \texttt{Event classification},
%  \texttt{Event narratives}}
%\ODPDEFINEDIN{DUL.owl}
%\ODPQUESTION{
%  \emph{Is there anyone responsible for the event?}
%  \emph{What are they trying to do?}
%  \emph{How did an event unfold?}}
%\ODPGRAPHIC{
%\begin{tikzpicture}
% \node[owlclass] (ACTION) {
% \begin{owlclass}{ACTION}
%  \item \texttt{Event}
%  \item $(\exists \emph{hasParticipant}.\texttt{Agent})$
% \end{owlclass}
% };
% \node[owlclass,below=0.6cm of ACTION] (PROCESS) {
% \begin{owlclass}{Process}
%  \item \texttt{Event}
% \end{owlclass}
% };
%\end{tikzpicture}
%}
%\end{ODP}

%\newpage


%\todo{Describe the OBJ Module of Soma}

%\subsection{Situations}

%\todo{Describe the EXEC Module of Soma}

%\subsubsection{Ontology Design Patterns (ODP)}


% In version \neemversion the \neemnar consists the belief state and action task hierarchy. 
% In the following sections we will describe how the belief state and action task hierarchy are represented.
% An concrete example for a logged \neemnar will be given in Chapter \ref{ch:example}.

% \subsection{Belief State}
% 
In EASE, we investigate everyday manipulation activities.
These involve the interaction with physical objects such as
grasping an object and putting it somewhere,
or taking a tool and applying it on an object to achieve a certain effect.
To tell a comprehensive story about its activity,
an agent needs to memorize the relation of action to
physical objects that are salient during the action.

One requirement for this is that beliefs of the agent about the existence
of physical objects must be maintained by the system.
In some cases, when the existence is known in advance,
it is sufficient to supply this knowledge to the agent through
a static ontology holding facts about existence of physical objects
in the environment of the agent.
This is, for example, useful to supply knowledge about a static
environment such as a kitchen with fixed set of appliances.
In more dynamic set-ups, however, the beliefs about the existence
of physical objects must be formed through specialized perception
methods, or through logical reasoning.
This is, for example, the case when the particular set of objects
contained in a drawer are unknown in advance.
We consider this type of information as part of the NEEM narrative
to keep these beliefs persistent, and to allow referring to
the perceived objects in action descriptions.

Episodic memories capture information about temporal situations
during which the beliefs of the agent evolve according to
perception, action, and logical inference.
NEEMs capture this evolution of beliefs.
As such, revised or false beliefs are not deleted but kept
persistently as part of the NEEM narrative.
This is to allow for deeper analysis of the agent performance,
and to provide richer input for learning algorithms.
This is realized through a 4D ontology that we describe at
the end of this section.

%%%%%%%%%%%%%%%%%%%%%%%%%%%%%
%%%%%%%%%%%%%%%%%%%%%%%%%%%%%
%%%%%%%%%%%%%%%%%%%%%%%%%%%%%
%%%%%%%%%%%%%%%%%%%%%%%%%%%%%
\subsection{Tangible Objects}
To refer to objects in action descriptions only the name of the 
object must be known to the NEEM acquisition system.
\knowrob comes with a rather comprehensive object type system,
with about XXX different object classes~\cite{knowrob-ontology}.
The type system is represented as part of the TBOX of the knowledge base.
Perceived objects are represented as instance of one of the object types
provided by the \knowrob system.
This type of information is part of the ABOX of the knowledge base.
The name of an object is usually the name of the class followed
by a underscore character and a 8-digit hash,
but could potentially be any unique name string.

\todo{introduce base class of objects, give some details about taxonomy}

In systems relying on sensor information and statistical models one has to deal
with the uncertainty coming from the use of these information sources.
This also affects beliefs about the identity of objects.
The problem of identity resolution can trivially be approached
by computing euclidean distance to known objects.
If the distance is below a certain threshold, it can often be assumed that it is the same object.
Background knowledge can be used to find better estimates for object identity:
An object is inside the gripper as long as it is grasped,
pulled down by gravity to the plane below when the grasp is released again,
and so forth.
Also, the identity of objects is often not so important for decision making,
but rather the properties of the object.
For successfully preparing a slice of bread, for example, any butter will do.
But if there are two butter packages and one is not yet opened one would rather
use the opened one to avoid it getting rancid.
When acquiring NEEMs one has to deal with this identity resolution problem
to allow for referring to objects in the narrative part of the episodic memory.

In the remainder of this section we describe the fundamental binary predicates to
describe physical objects in the belief state.

\begin{description}
\item[\textbf{rdf:type}] 
The most fundamental assertion about an object next to its name is its type.
Types are asserted through the \emph{rdf:type}
property with one of the object types as value.
Ontologies are subsumption hierarchies that derive 
knowledge about more specific concepts from more general concepts.
This allows, for example, to describe the relation between object and action
on a general level that applies to a larger class of objects and actions.
It is further possible to assert multiple types from different branches
of the type hierarchy for a single object.
For example, some mobile phone may also be used as projector while other mobile phones,
without the projector hardware, should not be classified as projector.
%%%%%%%%%%%%%%%%%%%%%%%%%%%%%
%%%%%%%%%%%%%%%%%%%%%%%%%%%%%
\item[\textbf{physicalParts}]
$Whole\ \emph{physicalParts}\ Part$ means
that $Part$ is tangible and one of the distinct parts of the
more complex tangible object $Whole$.
Manipulation is often about putting together parts to form a bigger whole.
During table setting, for example, objects are arranged such that they form
place settings for the different participants.
The objects are placed intentionally such that humans can easily reach them from their seat,
see which objects belong to which place setting, and infer what objects belong to whom.
Another example are assembly tasks during which an agent tries
to put together some mechanical parts using screwing connections, snap-in connection,
and so on to create an assembled product from scattered pieces available.
NEEMs can also represent this partonomy information through dedicated predicates
linking parts to the bigger whole.
In particular, the \ease ontology defines the following sub-properties of \emph{physicalParts}:
\emph{dangerousParts} (e.g., the blade of a knife),
\emph{electricalParts} (i.e., parts that need electrical current to work),
\emph{mealComponents} (i.e., the ingredients of a meal).
%%%%%%%%%%%%%%%%%%%%%%%%%%%%%
%%%%%%%%%%%%%%%%%%%%%%%%%%%%%
\item[\textbf{frameName}] \dots
\end{description}

\todo{some other relevant predicates? shape, color, mesh, pose?}

%%%%%%%%%%%%%%%%%%%%%%%%%%%%%
%%%%%%%%%%%%%%%%%%%%%%%%%%%%%
%%%%%%%%%%%%%%%%%%%%%%%%%%%%%
%%%%%%%%%%%%%%%%%%%%%%%%%%%%%
\subsection{Kinematic Objects}
\begin{enumerate}
 \item \todo{object parts can be connected through joints}
 \item \todo{what types of joints exist?}
 \item \todo{what predicates can be used to describe them?}
\end{enumerate}

%%%%%%%%%%%%%%%%%%%%%%%%%%%%%
%%%%%%%%%%%%%%%%%%%%%%%%%%%%%
%%%%%%%%%%%%%%%%%%%%%%%%%%%%%
%%%%%%%%%%%%%%%%%%%%%%%%%%%%%
\subsection{Environment Maps}
Semantic maps are detailed representations of the environment of some agent.
These include geometrical and visual information about objects and appliances,
their functional decomposition, and their state.
Modern game engines reach an impressive level of detail, with individual
leaves falling down trees, etc.
In EASE, we try to reach this level of detail in our semantic map representations.
We do this to provide very comprehensive information about situated experiences
from which general knowledge can be learned.
We believe that this high level of detail will allow us to learn more robust
models and with less training data then would be possible with
a more abstracted representation of environments.

The \emph{SemanticMap} itself could be a \emph{Room}, a \emph{Building}, or some other type of connected
region in which some agent can do navigation and manipulation.
The ontology defines some more specific room classes including
\emph{Kitchen}, \emph{LivingRoom} and \emph{StoreRoom}.
This is useful, for example, to correlate objects with their likely storage room,
or to relate the rooms with common activities performed in them.

\begin{description}
%%%%%%%%%%%%%%%%%%%%%%%%%%%%%
%%%%%%%%%%%%%%%%%%%%%%%%%%%%%
\item[\textbf{describedInMap}]
$Object\ \emph{describedInMap}\ Map$ means \dots
\end{description}

\todo{also describe how rooms relate to buildings, etc.?}
\todo{something about the state?}


\subsection{Temporal Representation}
\dots

% \subsection{Action Hierarchy}
% \label{ch:narrative,sec:actionHierarchy}
% In this section, we will describe how an action task hierarchy is represented in the \neemnar. 
% Since we are using \cram on our robots our plans do not necessary generate a sequence of actions, instead they will generate rather an hierarchy of actions.
% During the plan execution we are logging all executed actions with its parameters and represent the hierarchy in \owl.
% The general idea of the model is that an action will be represented as an individual of the class \owlClass{knowrob:'Action'}.
% This individual can be a direct instance of the class \owlClass{knowrob:'Action'} or its subclass.
% \todo{Add all supported action sub classes}
% With the predicates \owlPredicate{subAction}, \owlPredicate{previousAction} and \owlPredicate{nextAction}, which all have as subject and object the type \owlClass{knowrob:'Action'}, we are able to represent the action hierarchy.
% In our understanding an logged action hierarchy in an \owl represents all actions which were executed during an experiment.
% Meaning, if we would extracted all actions from the \owl file and recreated the action tree, we would be able to analyze and reasoning about all executed actions during one specific experiment.

% In the next subsections we will describe the predicates and classes which we currently defined in the version \neemversion to log the actions which were executed by the robot during an experiment. 

% \subsection{Action Predicates}
% Every individual of the class \owlClass{knowrob:'Action'} class or its subclass which will be logged in the \owl file can be asserted with the following predicates (see Table \ref{table:action_task_predicates}).
% Some predicates in the table are marked as required.
% This means that if you are intending to upload your \owl file to \openease, every \owlClass{knowrob:'Action'} individual has to have the required properties asserted.
% %Otherwise the \owl file will be revoked from the server.\todo{We have to implement such checking in \openease.}
% The \openease server checks also if the objects of the predicates are associated with the correct class.

% \begin{table}[H]
% 	\begin{tabular}{| c | c | c | c |}
% 		\hline			
% 		\textbf{Subject} & \textbf{Predicate} & \textbf{Object}  & \textbf{Required} \\
% 		\hline
% 		Action & taskSuccess & xsd:boolean & Yes \\
% 		\hline
% 		Action & startTime & Timepoint & Yes \\
% 		\hline
% 		Action & endTime & Timepoint  & Yes \\
% 		\hline
% 		Action & subAction & Action & No \\
% 		\hline
% 		Action & nextAction & Action & No \\
% 		\hline
% 		Action & previousAction & Action & No \\
% 		\hline
% 	\end{tabular}
% 	\caption{Action Predicates}
% 	\label{table:action_task_predicates}
% \end{table}

% \begin{description}
% 	\item[\textbf{taskSuccess}] 
% 		This predicates points to data type \owlClass{xsd:boolean}.
% 		The value \textbf{true} represents that the action was executed successfully.
% 		If any errors occurred during the action execution, the data type will be set to \textbf{false}.
% 		\footnote{In the NEEM version \neemversion we do not log the exact error which happened during action.}
% 	\item[\textbf{startTime}]
% 		The \owlPredicate{startTime} represents when the action started.
% 		Instead of representing the startTime as a data point, we are creating an instance of the class \owlClass{Timepoint}.
% 		The implemenation is done in that way because it made writing prolog queries much more convenient. 
% 		The name of the individual is representing the exact time when the action started e.g.\ \textit{timepoint\_1523878415038090} which can be understood that the action started 1523878415038090 microseconds after 00:00:00 UTC, Thursday, 1 January 1970.
% 		We are using the Unix time to represent a time point \cite{matthew2011beginning}.
% 		However, we are considering to measure the time in microseconds.
% 		The reason for this decision is that we also want to create NEEMs in simulation.
% 		However, tasks in simulation can be executed so fast that logging in microseconds allowed to measure the performance of the task executions.
% 		Also the measurement in microseconds allowed to differentiate the running time between tasks.
% 	\item[\textbf{endTime}] 
% 		This predicate represents when the action ended.
% 		More information about how we log time points is described in the predicate description \owlPredicate{startTime}.
% 	\item[\textbf{subAction}]
% 		The predicate \owlPredicate{subAction} allows to create parent-child relation between two tasks. In the context of this predicate, the subject is the parent action and the object is the child action.
% 		It is possible that an \owlClass{Action} instance can have multiple \owlPredicate{subAction} predicates which point each to a single child action.
% 	\item[\textbf{nextAction}]
% 		To be able to create an sequential order of actions which where executed on the same hierarchy level, we defined the \owlPredicate{nextAction} predicate.
% 		The subject represents the action which was started first and points to the next sibling actions.
% 		\footnote{In the NEEM version \neemversion we are not differentiate if actions were executed in sequence or in parallel.}
% 	\item[\textbf{previousAction}]
% 		Like in \owlPredicate{previousAction} this predicate is created to create an sequential order between siblings tasks.
% 		However in this case, \owlPredicate{previousAction} connects an \owlClass{Action} instance with the sibling action which was performed previously.
% \end{description}


% \subsection{Action Parameter Predicates}
% 	\label{sec:actionParameterPredicactes}
% 	In general, actions have to have parameters which have to be asserted.
% 	Based on those assertions \cram is able to infer how to perform the task.
% 	For instance, a grasping task will be executed differently when the target object is a spoon compared to when the target object is a bottle.
% 	To understand the logged behavior of the robot better, we are logging to each action the corresponding parameters.
% 	For \cram we are using for our actions a set of predefined parameters.
% 	For example, when an action requires an object every \cram action has parameter name \textit{object} asserted.
% 	During the logging process, we are creating based on the parameter's value an instance of the class \owlClass{Object} and connect this instance with the \owlPredicate{objectActedOn}.
% 	\todo{Ref to belief state when object description is done}
% 	This implementation allows us to use the logger for new actions without the need to extend the logger.
% 	As long the \cram actions will use the predefined parameter names all parameters will be logged without the need to extend the logger.
% 	All predefined parameters are represented by a separated predicates which point to the parameter value which is an individual of the corresponding \owl class.
% 	Table \ref{table:action_parameter_predicates} shows the current parameter predicates which are supported by our NEEM representation.
% 	The design of the action parameter predicates is based on the work with our \pr.
% 	Therefore in the NEEM \neemversion it might be possible that the action parameters cannot be used by everyone in this state.
	 
% \begin{table}[H]
% \begin{tabular}{| c | c | c |}
% 	\hline			
% 	\textbf{Subject} & \textbf{Predicate} & \textbf{Object} \\
% 	\hline			
% 	Action & effort & qudt\#NewtonMeter \\
% 	\hline
%     Action & position & Float \\
%     \hline
% 	Action & arm & Pr2\#Pr2RightArm \\
% 	\hline
% 	Action & bodyPartsUsed & Pr2\#Pr2RightGripper \\
% 	\hline
% 	Action & goalLocation & Pose or Connected Space Region \\
% 	\hline
%     Action & objectActedOn & Object \\
% 	\hline
% 	Action & objectType & Object \\
% 	\hline
% \end{tabular}
% 	\caption{Action Parameter Predicates}
% 	\label{table:action_parameter_predicates}
% \end{table}

% \begin{description}
% 	\item[\textbf{effort}] 
% 		Effort is the grasping force in newton-meters.
% 		To model this we are using the \qudt ontology \footnote{http://qudt.org/}.
% 	\item[\textbf{position}]
% 		We are using this predicate to log the the goal position for the gripper of the \pr.
% 		The \pr accepts a joint angle in RAD to position its gripper.
% 		We decided to use a float data type to model the position to be able to represent more different types of position.
% 		For instance, our \boxy robot uses centimeters to position the gripper.
% 		Therefore with a float representation we are able to log from both robots the position parameter.
% 	\item[\textbf{arm}]
% 		With this predicate we want to log which arm was used to by the robot.
% 		The predicate points to an instance of the specific robot arm class.
% 		For instance, to model that the \pr used a right arm to grasp a bottle we are asserting the \owlPredicate{arm} predicate to an individual of the class \owlClass{Pr2RightArm}\footnote{http://knowrob.org/kb/PR2.owl}.
% 	\item[\textbf{bodyPartsUsed}]
% 		The \owlPredicate{bodyPartsUsed} predicate should represent which body part from the robot was used to perform the task.
% 		This predicate can be used \eg to represent that a gripper was used to perform the task.
% 		Like in \dots we also want to represent the body part as instance of the class which represents the body part.
% 	\item[\textbf{goalLocation}]
% 		This predicate can be used to represent the target or location parameter of an action.
% 		This predicate is suitable to represent \eg the target location of an \textit{Going} action.
% 		We are considering to different object types to log the goal location.
% 		Since our robots are build on \ros our robots can work with \textit{Poses} \footnote{http://docs.ros.org/api/geometry\_msgs/html/msg/Pose.html}.
% 		How we represent this \owl class is stated out in section \ref{sec:pose}.
		
% 		Since we are also using \cram, our robots can also encounter \textit{location designator} as goal locations\cite{beetz2010cram}.
% 		That means \cram can handle tasks like "Go to the kitchen counter".
% 		Since locations designators are more abstract we use the \owl class \owlClass{Connected Space Region} to log them.
% 		A more detailed description is given in section \ref{sec:connectedSpaceRegion}.
		
% 	\item[\textbf{objectActedOn}]
% 		With the predicate \owlClass{objectActedOn} we want to log which objects are given as parameter to the action.
% 		For instance, we can log which objects the robots looked for during a perception task and what objects he tried to grab.
% 	\item[\textbf{objectType}]
% 		The difference of \owlPredicate{objectType} compared to \owlPredicate{objectActedOn} is that \owlPredicate{objectType} define not a specific object but rather an object in general.
% 		For instance, if the robotic agent will get a task such as "grab the milk from the fridge".
% 		Given the task, the agent knows only that it has to grab a object of the type of milk from the fridge.
% 		So at this moment the robot does not know if the a milk box is actually in the fridge.
% 		Thereofre it has to recognize the milk first.
% 		If this was sucessfull then the robot will assoicated the following task the milk ID with the predicate \owlPredicate{objectActedOn} which the object will be the link to the speicifc object in the belieft state.
% \end{description}

% \subsection{Action Parameter Classes}
% As shown in section \ref{sec:actionParameterPredicactes} we are not only using data types to log the action parameter.
% In this section we are describing the \owl classes which we introduced to be able to log all parameters which we required to log during our experiments.

% \subsubsection{Pose}
% 	\label{sec:pose}
% 	This class is used to log coordinates given as parameter.
% 	Since our robots are using \ros we are logging poses.
% 	A pose consists from a Quaternion and a 3D vector.
% 	Since both together can only describe a Pose, both entities are required to be logged.
% 	\begin{table}[H]
% 		\begin{tabular}{| c | c | c | c |}
% 			\hline			
% 			\textbf{Subject} & \textbf{Predicate} & \textbf{Object} & \textbf{Required}\\
% 			\hline
% 			Pose & quaternion & String & Yes \\
% 			\hline
% 			Pose & translation & String & Yes \\			
% 			\hline
% 		\end{tabular}
% 		\caption{Pose Predicates}
% 		\label{table:pose_predicates}
% 	\end{table}
	
% 	\begin{description}
% 		\item[Quaternion] 
% 			%Quaternions are used to describe rotations in a three dimensionally space. \todo{Set reference to a Quaternions description}
% 			In the NEEM version \neemversion we are representing quaternions as a string.
% 			For instance, the quaternion $0.5 + 0.35i + 1j +0k$ will be represented as "0.5 0.35 1 0".
% 		\item[Translation]
% 			The \owlClass{Translation} class represents the 3D vector part of the pose.
% 			In the NEEM version \neemversion we are representing vectors as a string.
% 			For instance, the vector $\begin{bmatrix} -0.759, 1.19, 0.932 \end{bmatrix}^T$ will be represented as "-0.759 1.19 0.932".
			
			



% 	\end{description}
	
	
	
% \subsubsection{Connected Space Region}
% 	\label{sec:connectedSpaceRegion}
% 	Since we are using \cram our robots are allowed to use location designators to describe location parameters which will be resolved to pose \cite{beetz2010cram}
% 	The resolving process is divided in three stages:
% 		\begin{enumerate}
% 			\item Define a abstract location or general location. For instance, "Grab the milk from the fridge.".
% 			\item \cram will try out to resolve "fridge" to an entity of the semantic map.
% 			\item The last step is to resolve the entity from the semantic map into a pose with which the robot can actually work with.
% 		\end{enumerate}
	
% 	\begin{table}[H]
% 		\begin{tabular}{| c | c | c | c |}
% 			\hline			
% 			\textbf{Subject} & \textbf{Predicate} & \textbf{Object} & \textbf{Required}\\
% 			\hline
% 			Connected Space Region & onPhysical & iai-kitchen & No \\
% 			\hline
% 		\end{tabular}
% 		\caption{Connected Space Region Predicates}
% 		\label{table:connected_space_region_predicates}
% 	\end{table}
	
% 	\begin{description}
% 		\item[onPhysical] 
% 			To be able to represent the resolution to an entity of the semantic map we defined the \owlPredicate{onPhysical} predicate.
% 			This predicate points to an instance of semantic map.
% 			For instance, to grasp an object from the kitchen counter in our kitchen we link the \owlClass{ConnectedSpaceRegion} instance to an individual of our semantic mal in this case \textit{knowrob:iai\_kitchen\_sink\_area\_counter\_top}.
% 	\end{description}

% \subsubsection{Timepoint}
% 	In the NEEM version \neemversion the \owlClass{Timepoint} does not have any predicates.
% 	An instance of the \owlClass{Timepoint} class defines a moment in time which is represented in microseconds.
% 	The exact timestamp in microseconds is represent in the name of the instance.
% 	For example, the individual with the name "timepoint\_1523878419.243441" represents a timepoint which is 1523878419243441 microseconds after 00:00:00 UTC, Thursday, 1 January 1970.
% 	We are using the Unix time to represent a time point\cite{matthew2011beginning}.
% 	This Timepoint represenation makes the prolog quering much more easier.

% 
% % % % % % % % % % % % % % % % % % % % % % % %
% % % Prelude
% % % % % % % % % % % % % % % % % % % % % % % %
\newcommand{\givenODPNAME}{}
\newcommand{\givenODPINTENT}{}
\newcommand{\givenODPDEFINEDIN}{}
\newcommand{\givenODPDESCRIPTION}{}
\newcommand{\givenODPGRAPHIC}{}
\newcommand{\givenODPDOMAIN}{}
\newcommand{\givenODPQUESTION}{}
\newcommand{\ODPINTENT}[1]     {\renewcommand{\givenODPINTENT}{#1}}
\newcommand{\ODPDEFINEDIN}[1]  {\renewcommand{\givenODPDEFINEDIN}{#1}}
\newcommand{\ODPDESCRIPTION}[1]{\renewcommand{\givenODPDESCRIPTION}{#1}}
\newcommand{\ODPGRAPHIC}[1]    {\renewcommand{\givenODPGRAPHIC}{#1}}
\newcommand{\ODPDOMAIN}[1]     {\renewcommand{\givenODPDOMAIN}{#1}}
\newcommand{\ODPQUESTION}[1]   {\renewcommand{\givenODPQUESTION}{#1}}
\newcommand{\OPDinit}{
  \renewcommand{\givenODPINTENT}{REQUIRED!}
  \renewcommand{\givenODPDEFINEDIN}{REQUIRED!}
  \renewcommand{\givenODPDESCRIPTION}{REQUIRED!}
  \renewcommand{\givenODPGRAPHIC}{REQUIRED!}
  \renewcommand{\givenODPQUESTION}{}
  \renewcommand{\givenODPDOMAIN}{}
  \renewcommand{\labelitemi}{$\mathbf{\sqsubseteq}$}
}

\newenvironment{owlclass}[2][,] {
  \begin{minipage}{5.0cm}
  \begin{center}
  \texttt{\bf#2} \\[-0.2cm]
  \par\noindent\rule{\textwidth}{0.4pt}
  \vspace{-0.6cm}
  \begin{itemize}[#1]
  \raggedright} {
  % % % % % %
  \end{itemize}
  \end{center}
  \end{minipage}
}

\newenvironment{ODP}[1]{
\OPDinit
\renewcommand{\givenODPNAME}{#1}
}{
\givenODPDESCRIPTION
\begin{figure}[htb]
\begin{minipage}{0.45\textwidth}
\begin{tabular}{ p{1.8cm} p{3.2cm} }
\toprule
% {\it\bf Name}                 & \emph{\givenODPNAME} \\
{\it\bf Intent}               & \givenODPINTENT \\
{\it\bf Domains}              & \givenODPDOMAIN \\
{\it\bf Competency Questions} & \givenODPQUESTION \\
{\it\bf Defined in}           & \givenODPDEFINEDIN \\
\bottomrule
\end{tabular}
\end{minipage}
\begin{minipage}{0.55\textwidth}
\begin{center}
\givenODPGRAPHIC
\end{center}
\end{minipage}
\caption{The \emph{\givenODPNAME} ODP.}
\end{figure}
}

\tikzset{owlclass/.style={draw=blue!40,fill=blue!20,rounded corners}}

% % % % % % % % % % % % % % % % % % % % % % % %
% % % % % % % % % % % % % % % % % % % % % % % %
% % % % % % % % % % % % % % % % % % % % % % % %
\chapter{Ontology Design Patterns (ODPs)}

% % % % % % % % % % % % % % % % % % % % % % % %
% \section{Object Roles}

% % % % % % % % % % % % % % % % % % % % % % % %
% % % % % % % % % % % % % % % % % % % % % % % %
% \section{Quality and Quality Regions}

% % % % % % % % % % % % % % % % % % % % % % % %
% % % % % % % % % % % % % % % % % % % % % % % %
% \section{Designed Artifacts}

% % % % % % % % % % % % % % % % % % % % % % % %
% % % % % % % % % % % % % % % % % % % % % % % %
\section{Task Execution ODP}
\begin{ODP}{Task Execution}
\ODPDESCRIPTION{
This ODP allows to make assertions on roles played by agents
without involving the agents that play that roles, and vice versa.
It allows to express neither the context type in which tasks are defined,
not the particular context in which the action is carried out.
Moreover, it does not allow to express the time at which
the task is executed through the action
(for actions that do not solely execute that certain task).}
\ODPINTENT{To represent actions through which tasks are executed. }
\ODPDOMAIN{
  \texttt{Organization},
  \texttt{Management},
  \texttt{Scheduling},
  \texttt{Workflow}}
\ODPDEFINEDIN{DUL.owl}
\ODPQUESTION{
  \emph{Which task is executed through this action?}
  \emph{What actions can execute that task?}}
\ODPGRAPHIC{
\begin{tikzpicture}
 \node[owlclass] (ACTION) {
 \begin{owlclass}{Action}
  \item $(\geq 1\ \emph{executesTask}.\texttt{Task})$
 \end{owlclass}
 };
 \node[owlclass,below=0.6cm of ACTION] (TASK) {
 \begin{owlclass}{Task}
  \item $(\forall \emph{isExecutedIn}.\texttt{Action})$
 \end{owlclass}
 };
 \draw (ACTION) edge[thick,-,dashed,blue!60] (TASK);
\end{tikzpicture}
}
\end{ODP}

\newpage
\section{Quality -- Region ODP}
\begin{ODP}{Quality -- Region}
\ODPDESCRIPTION{This ODP allows structuring information about the properties of an Entity. It distinguishes between dependent aspects of the Entity (things that cannot exist without the Entity itself existing), and the values that may be ascribed to those aspects. These values may be points in some Region. Note that a Region may be a finite set of discrete labels, allowing for ``qualitative'' descriptions, but more often a Region is some dimensional space allowing ``quantitative'' descriptions. A Region may contain a single point, in cases where the value of a Quality is known precisely.}
\ODPINTENT{To distinguish between an aspect of an Entity and a particular numerical description of it.}
\ODPDOMAIN{
  \texttt{Measurement},
  \texttt{Object representation},
  \texttt{Environment representation},
  \texttt{Execution status}}
\ODPDEFINEDIN{DUL.owl}
\ODPQUESTION{
  \emph{What qualities does an entity have?}
  \emph{What are possible values for a quality?}
  \emph{What is the actual value of a quality for a particular entity (at a particular time)?}}
\ODPGRAPHIC{
\begin{tikzpicture}
 \node[owlclass] (QUALITY) {
 \begin{owlclass}{Quality}
  \item $(\exists \emph{isQualityOf}.\texttt{Entity})$
  \item $(\exists \emph{hasRegion}.\texttt{Region})$
 \end{owlclass}
 };
 \node[owlclass,below=0.6cm of QUALITY] (REGION) {
 \begin{owlclass}{Region}
  \item $(\exists \emph{isRegionFor}.\texttt{Quality})$
 \end{owlclass}
 };
 \draw (QUALITY) edge[thick,-,dashed,blue!60] (REGION);
\end{tikzpicture}
}
\end{ODP}

\newpage
\section{Process vs. Action ODP}
\begin{ODP}{Process vs. Action}
\ODPDESCRIPTION{An Action is an Event with at least one Agent participant, such that this Agent has a Task, often defined by a Plan or Workflow, which it executes through the Action. A Process is an Event for which no such commitments have been made. In DUL, these classes are not disjoint, allowing a particular event individual to be classified as either, depending on whether we care to record an agent and its goals or not. In EASE, we use Process as a top-level class for events with no agentive participant.}
\ODPINTENT{To represent the intentional and agentive structure-- or lack thereof-- behind Events.}
\ODPDOMAIN{
  \texttt{Event classification},
  \texttt{Event narratives}}
\ODPDEFINEDIN{DUL.owl}
\ODPQUESTION{
  \emph{Is there anyone responsible for the event?}
  \emph{What are they trying to do?}
  \emph{How did an event unfold?}}
\ODPGRAPHIC{
\begin{tikzpicture}
 \node[owlclass] (Action) {
 \begin{owlclass}{Action}
  \item \texttt{Event}
  \item $(\exists \emph{hasParticipant}.\texttt{Agent})$
 \end{owlclass}
 };
 \node[owlclass,below=0.6cm of ACTION] (PROCESS) {
 \begin{owlclass}{Process}
  \item \texttt{Event}
 \end{owlclass}
 };
\end{tikzpicture}
}
\end{ODP}

\newpage
\section{Designed Artifact ODP}
\begin{ODP}{Designed Artifact}
\ODPDESCRIPTION{A DesignedArtifact is a physical object described by a Design. In EASE, Designs refer to the form, but also the function of an object. This allows us to say that an object is ``for'' a particular purpose, even though it might be used for something else instead. For example, a cup is a BeverageContainer but can be used as a Flowerpot. Designs form a hierarchy of specificity, for example $DesignMilkContainer \sqsubseteq DesignBeverageContainer \sqsubseteq DesginContainer \sqsubseteq Design$. The justification for this pattern is that the type of an object is rigid, but the roles it plays in events change. A naive taxonomy, without a notion similar to Design, cannot tackle the fact that objects are usable in several ways beyond the obvious; a hammer isn't always a hammer, sometimes it's a paperweight. On the other hand, a usable ontology of objects must take into account how human users refer to objects by their default use.}
\ODPINTENT{To explicate the intuitive classification human users would have of objects, based on their default uses.}
\ODPDOMAIN{
  \texttt{Object classification},
  \texttt{Event narratives}}
\ODPDEFINEDIN{DUL.owl, EASE.owl, EASE-middle.owl}
\ODPQUESTION{
  \emph{What sort of object is this?}
  \emph{What is the intended use of the object?}
  \emph{How did an event unfold?}}
\ODPGRAPHIC{
\begin{tikzpicture}
 \node[owlclass] (DESIGNEDARTIFACT) {
 \begin{owlclass}{DesignedArtifact}
  \item \texttt{PhysicalArtifact}
  \item $(\exists\emph{isDescribedBy}.\texttt{Design})$
 \end{owlclass}
 };
 \node[owlclass,below=0.6cm of DESIGNEDARTIFACT] (DESIGN) {
 \begin{owlclass}{Design}
  \item \texttt{Description}
 \end{owlclass}
 };
 \draw (DESIGNEDARTIFACT) edge[thick,-,dashed,blue!60] (DESIGN);
\end{tikzpicture}
}
\end{ODP}

\newpage
\section{Transient ODP}
\begin{ODP}{Transient}
\ODPDESCRIPTION{This pattern addresses the fact that objects change by undergoing/taking part in Processes. For example, the PancakeMix becomes, through Baking, a Pancake. However, the ontological status of the object while the process takes place is unclear: the object placed on the frying pan is not a Pancake until Baking finishes, but it's not PancakeMix either once it begins to coagulate. In the EASE approach, the object in-between such characterizations is a Transient. A Transient transitionsFrom an Object, and possibly transitionsTo an Object. It may also be the case that a Transient transitionsBack to an Object, to indicate that once the process completes, the same Object is restored; this would be the case for example for catalysts in chemistry, or a loaf of bread after slicing, if there is enough bread left.}
\ODPINTENT{Ontological classification for objects undergoing type changes.}
\ODPDOMAIN{
  \texttt{Object classification},
  \texttt{Event narratives}}
\ODPDEFINEDIN{EASE.owl, EASE-middle.owl}
\ODPQUESTION{
  \emph{What sort of object is this?}
  \emph{What objects ``went'' in the making of another?}
  \emph{Does an object preserve or restore its identity after change?}}
\ODPGRAPHIC{
\begin{tikzpicture}
 \node[owlclass] (TRANSIENT) {
 \begin{owlclass}{Transient}
  \item \texttt{Object}
  \item $(\exists\emph{transitionsFrom}.\texttt{Object})$
  \item $(\forall\emph{transitionsTo}.\texttt{Object})$
 \end{owlclass}
 };
 \node[owlclass,below=0.6cm of TRANSIENT] (TRANSITIONSBACK) {
 \begin{owlclass}{transitionsBack}
  \item \texttt{transitionsFrom}
  \item \texttt{transitionsTo}
 \end{owlclass}
 };
\end{tikzpicture}
}
\end{ODP}

\newpage
\section{States}
\begin{ODP}{State, Configuration, Gestallt}
\ODPDESCRIPTION{A state is a configuration of the world that is construed to be stable on its own. Outside disturbances may cause state transitions, and the settling into some other, self-stable configuration. A State is also characterized by a Description, that indicates things such as what kind of entities participate in the state, what relations might exist between them, what regions may be used by particular qualities of the participants. This Description is, in general, referred to as Gestallt, however some common examples are Goals-- describe desired states of the world--, Norms-- describe states that should be kept--, and Diagnoses-- describe a state that causes certain observable symptoms.}
\ODPINTENT{Ontological representation for situations in the world that are cognitively construed as stable arrangements of entities.}
\ODPDOMAIN{
  \texttt{Event classification},
  \texttt{Event narratives}}
\ODPDEFINEDIN{EASE.owl, EASE-middle.owl}
\ODPQUESTION{
  \emph{What are stable arrangements?}
  \emph{What is meant by ``state'' of the world?}
  \emph{What characterizes a state?}}
\emph{Examples}
\begin{itemize}
  \item AssemblyConnection: two objects are in a rigid connection, such that the movement of one determines the movement of the other. In this case the characterizing Gestallt for this State uses several Roles-- one for each part/geometric feature belonging to the connected objects-- and puts constraints on the relative positioning of these geometric features such that they interlock to produce the rigid connection.
  \item Contact: two objects are in mechanical contact. The characterizing Gestallt uses two Roles, one for each participating object, and puts constraints on the Pose qualities of the participants: the poses should be such that the participants touch.
  \item FunctionalControl: an object restricts the movement of another, at least partially. The Gestallt uses the Roles Item and Restrictor. More concrete examples are Containment: the Restrictor is a Container, and the Pose quality of the Item should use the region inside the Container; and Support: both Restrictor and Item are objects, placed in such a way that the Item does not move because of gravity.
  \item PhysicallyAccessible: the Gestallt for this state uses the roles Item, a Container or Protector, and optionally an Accessor and a Task, and states that an Item is placed either in a Container or protected by a Protector, but the placement of the Item and Container is such that an Accessor may nevertheless reach the Item in order to perform a Task. For a more concrete example, a DoorOpen is a kind of PhysicallyAccessible where the Protector is a door, the Item is the inside of the room behind the door, the Accessor is some person and the Task is to walk into the inside of the room.
\end{itemize}
%\ODPGRAPHIC{
%\begin{tikzpicture}
% \node[owlclass] (TRANSIENT) {
% \begin{owlclass}{Transient}
%  \item \texttt{Object}
%  \item $(\exists\emph{transitionsFrom}.\texttt{Object})$
%  \item $(\forall\emph{transitionsTo}.\texttt{Object})$
% \end{owlclass}
% };
% \node[owlclass,below=0.6cm of TRANSIENT] (TRANSITIONSBACK) {
% \begin{owlclass}{transitionsBack}
%  \item \texttt{transitionsFrom}
%  \item \texttt{transitionsTo}
% \end{owlclass}
% };
%\end{tikzpicture}
%}
\end{ODP}

% % % % % % % % % % % % % % % % % % % % % % % %
% % % % % % % % % % % % % % % % % % % % % % % %
% \section{Workflows}

% % % % % % % % % % % % % % % % % % % % % % % %
% % % % % % % % % % % % % % % % % % % % % % % %
% \section{Action Phases}

% % % % % % % % % % % % % % % % % % % % % % % %
% % % % % % % % % % % % % % % % % % % % % % % %
% \section{Transients}


% \subsection{Knowledge Bases}

In this chapter we should describe all knowledge bases we are using to represent the NEEMs.

\subsubsection{IAI-Kitchen}
\subsubsection{Knowrob}
\subsubsection{PR2}
\subsubsection{Qudt}

